% $Author$
% $Date$
% $Revision$

% HISTORY:
% 2006-10-31 - Oscar started
% 2007-08-19 - Stef revised
% 2007-11-09 - Andrew corrections
% 2008-03-28 - Cassou corrections
% 2009-07-07 - Oscar fixed broken tests

%=================================================================
%=================================================================
\ifx\wholebook\relax\else
% --------------------------------------------
% Lulu:
        \documentclass[a4paper,10pt,twoside]{book}
        \usepackage[
                papersize={6.13in,9.21in},
                hmargin={.75in,.75in},
                vmargin={.75in,1in},
                ignoreheadfoot
        ]{geometry}
        \input{../common.tex}
        \pagestyle{headings}
        \setboolean{lulu}{true}
% --------------------------------------------
% A4:
%       \documentclass[a4paper,11pt,twoside]{book}
%       \input{../common.tex}
%       \usepackage{a4wide}
% --------------------------------------------
    \graphicspath{{figures/} {../figures/}}
        \begin{document}
        % \renewcommand{\nnbb}[2]{} % Disable editorial comments
        \sloppy
\fi
%=================================================================
\chapter{Clases B\'asicas}

La mayor\'ia de la magia de Smalltalk no est\'a tanto en el lenguaje como en las librer\'ias de clases. Para programar efectivamente con Smalltalk, necesita aprender cómo la librer\'ia de clases soporta el lenguaje y el entorno. Toda la liber\'ia de clases a sido enteramente escrita en Smalltalk y puede ser facilmente ampliada desde un paquete que agregue alguna nueva funcionalidad a una clase si esta no se encuentra definida en ella.

Nuestro objetivo no es presentar en un tedioso detalle toda la librer\'ia de clases de \pharo, pero si resaltar las clases clave y los m\'etodos que necesita conocer para usar o evitar al programar efectivamente. En este cap\'itulo cubriremos las cases b\'asicas que necesitar\'a en cualquier aplicaci\'on: \ct{Object}, \ct{Number} and its subclasses, \ct{Character}, \ct{String}, \ct{Symbol} and \ct{Boolean}.

%=============================================================
\ifx\wholebook\relax\else
   \bibliographystyle{jurabib}
   \nobibliography{scg}
   \end{document}
\fi
%=============================================================

%-----------------------------------------------------------------

%%% Local Variables:
%%% coding: utf-8
%%% mode: latex
%%% TeX-master: t
%%% TeX-PDF-mode: t
%%% ispell-local-dictionary: "english"
%%% End: