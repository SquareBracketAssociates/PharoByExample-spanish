% $Author$
% $Date$
% $Revision$

% HISTORY:
% 2007-06-06 - Stef started
% 2007-08-21 - Oscar edit
% 2007-09-06 - Lukas corrections
% 2007-09-11 - Orla corrections
% 2009-12-18 - Pablo started spanish translation

%=================================================================
\ifx\wholebook\relax\else
% --------------------------------------------
% Lulu:
	\documentclass[a4paper,10pt,twoside]{book}
	\usepackage[
		papersize={6.13in,9.21in},
		hmargin={.75in,.75in},
		vmargin={.75in,1in},
		ignoreheadfoot
	]{geometry}
	\input{../common.tex}
	\pagestyle{headings}
	\setboolean{lulu}{true}
% --------------------------------------------
% A4:
%	\documentclass[a4paper,11pt,twoside]{book}
%	\input{../common.tex}
%	\usepackage{a4wide}
% --------------------------------------------
    \graphicspath{{figures/} {../figures/}}
	\begin{document}
	\renewcommand{\nnbb}[2]{} % Disable editorial comments
	\sloppy
\fi
%=================================================================
\chapter{Comprendiendo la sintaxis de los mensajes}
\chalabel{understanding}

Aunque la sintaxis de los mensajes de \st es extremadamente simple, no es convencional y puede llevar alg\'un tiempo sentirse familiarizado con ella.
Este cap\'itulo ofrece alguna gu\'ia para ayudarte a estar aclimatado a esta sintaxis especial de env\'io de mensajes.
Si ya te sientes c\'omodo con la sintaxis, puedes elegir saltear este cap\'itulo, o volver al mismo mas tarde.

%=============================================================
\section{Identificando mensajes}

En \st, excepto para los elementos sint\'acticos listados en \charef{syntax} (\ct+:= ^ . ; # () {} [ : | ]+), todo es un env\'io de mensaje.
Como en \ind{C++}, puedes definir operadores como \ct{+} para tus propias clases, pero todos los operadores tienen la misma precedencia.
Adem\'as, no puedes cambiar la aridad de un m\'etodo. ``\ct{-}'' es siempre un mensaje binario; no hay ninguna manera de tener un `` \ct{-}'' unario con una sobrecarga diferente.

En \st el orden en el cual los mensajes se env\'ian es determinado por el tipo de mensaje. Hay solo tres tipos de mensajes: mensajes \emphsubind{mensaje}{unarios}, mensajes \emphsubind{mensaje}{binarios}, y mensajes \emphsubind{mensaje}{keyword}. Los mensajes unarios siempre se env\'ian primero, luego los mensajes binarios y finalmente los mensajes keyword. Como en la mayor\'ia de los lenguajes, los \ind{par\'entesis} pueden ser utilizados para cambiar el orden de evaluaci\'on. Estas reglas hacen el c\'odigo \st lo mas f\'acil de leer posible. Y en la mayor parte del tiempo no tienes que estar pensando en las reglas.

Como la mayor parte del c\'omputo en \st se realiza a trav\'es del pasaje de mensajes, identificar correctamente los mismos es crucial. La siguiente terminolog\'ia va a ayudarnos:

\begin{itemize}
  \item Un mensaje se compone con un \emphsubind{mensaje}{selector} de mensaje y opcionalmente con argumentos de mensaje.
  \item Un mensaje es enviado a un \emphsubind{mensaje}{receptor}.
  \item La combinaci\'on de un mensaje y su receptor es denominada \emphsubind{mensaje}{env\'io} \emph{de mensaje} como se muestra en \figref{firstScriptMessage}.
%  \item We call expression a message send, a variable, an assignment, a literal or a block.
\end{itemize}

\begin{figure}[htb]
\begin{minipage}{0.53\textwidth}
	\begin{center}
	\includegraphics[width=0.95\textwidth]{message}
	\caption{Dos mensajes compuestos por un receptor, un selector de m\'etodo, y un conjunto de argumentos.\figlabel{firstScriptMessage}}\end{center}
\end{minipage}
\hfill
\begin{minipage}{0.43\textwidth}
	\begin{center}
	\ifluluelse
		{\includegraphics[width=0.9\textwidth]{uKeyUnOne}}
		{\includegraphics[width=6cm]{uKeyUnOne}}
    \caption{\ct{aMorph color: Color yellow} est\'a compuesto por dos env\'ios de mensaje: \ct{Color yellow} y \ct{aMorph color: Color yellow}.\figlabel{ellipse}}
	\end{center}
\end{minipage}
\end{figure}

%\begin{figure}[ht]
%\begin{center}
%\includegraphics[width=0.5\textwidth]{message}
%\caption{Two messages composed of a receiver, a method selector, and a set of arguments.\figlabel{firstScriptMessage}}\end{center}
%\end{figure}

\important{Un mensaje siempre es enviado a un receptor, el cual puede ser un simple literal, un bloque, una variable o el resultado de evaluar otro mensaje.}
%sd-this is not totally right because we do not cover assignment but ok for now

Para ayudarte a identificar el receptor de un mensaje, te lo subrayaremos.
Tambi\'en vamos a rodear cada env\'io de mensaje con una elipse con el n\'umero de env\'io de mensaje, comenzando por el primero que ser\'a enviado, para ayudarte a ver el orden en el cual los mensajes son enviados. 

%\begin{figure}[!ht]
%\begin{center}
%\includegraphics[width=6cm]{uKeyUnOne}
%\end{center}
%\caption{\ct{aMorph color: Color yellow} is composed of two expressions: \ct{Color yellow} and \ct{aMorph color: Color yellow}.\figlabel{ellipse}}
%\end{figure}

La \figref{ellipse} presenta dos env\'ios de mensajes, \ct{Color yellow} y \ct{aMorph color: Color yellow}, por lo tanto hay dos elipses. El env\'io de mensaje \ct{Color yellow} es ejecutado primero y en consecuencia su elipse es numerada con \ct{1}.  Hay dos receptores: \ct{aMorph} el cual recibe el mensaje \ct{color: ...} y \ct{Color} el cual recibe el mensaje \ct{yellow}. Ambos receptores est\'an subrayados.

Un receptor puede ser el primer elemento de un mensaje, como \ct{100} en el env\'io de mensaje \ct{100 + 200} o \ct{Color} en el env\'io de mensaje \ct{Color yellow}. De cualquier manera, un receptor puede ser tamb\'ien el resultado de otro env\'io de mensaje. Por ejemplo, en el mensaje \ct{Pen new go: 100}, el receptor del mensaje \ct{go: 100} es el objeto devuelto por el env\'io del mensaje \ct{Pen new}. En todos los casos, un mensaje es enviado a un objeto llamado \emph{receptor} el cual puede ser el resultado de otro env\'io de mensaje.

\begin{table}\centering
	\begin{tabularx}{\linewidth}{llX}
		\toprule
		Env\'io de mensaje & Tipo de mensaje & Resultado \\
		\midrule
		\lct{Color yellow}
			& unario
			& Crea un color.
		\\
		\lct{aPen  go: 100.}
			& keyword
			& El l\'apiz receptor se mueve 100 p\'ixeles adelante.
		\\
		\lct{100 + 20}
			& binario
			& El n\'umero 100 recibe el mensaje + con el n\'umero 20.
		\\
		\lct{Browser open}
			& unario
			& Abre un nuevo navegador.
		\\
		\lct{Pen new  go: 100}
			& unario y keyword
			& Se crea un l\'apiz y se lo mueve 100 p\'ixeles.
		\\
		\lct{aPen go: 100 + 20}
			& keyword y binario
			& El l\'apiz receptor se mueve 120 p\'ixeles adelante.
		\\
		\bottomrule
	\end{tabularx}
	\caption{Ejemplos de Env\'ios de Mensajes y sus Tipos}\tablabel{messageExamples}
\end{table}

La \tabref{messageExamples} muestra varios ejemplos de env\'ios de mensajes.
Deber\'ias notar que no todos los env\'ios de mensaje tienen argumentos. Los mensajes unarios como \ct{open} no tienen argumentos. Los mensajes keyword simples y los mensajes binarios como \ct{go: 100} y \ct{+ 20} tienen un argumento. 
Tambi\'en hay mensajes simples y compuestos. \ct{Color yellow} y \ct{100 + 20} son simples:  un mensaje es enviado a un objeto, mientras que el env\'io de mensaje \ct{aPen go: 100 + 20} est\'a compuesto por dos mensajes: \ct{+ 20} es enviado a \ct{100} y \ct{go:} es enviado a \ct{aPen} con el argumento que haya resultado del primer mensaje.
Un receptor puede ser una expresi\'on (como una asignaci\'on, un env\'io de mensaje o un literal) que devuelva un objeto. En \ct{Pen new go: 100}, el mensaje \ct{go: 100} es enviado al objeto que resulta de la ejecuci\'on del env\'io del mensaje \ct{Pen new}.


%=============================================================
\section{Tres tipos de mensajes}

\st define unas pocas reglas simples para determinar el orden en el cual los mensajes son enviados. Estas reglas est\'an basadas en la distinci\'on entre 3 tipos diferentes de mensajes: 
\begin{itemize}
\item Los \emph{Mensajes Unarios} son mensajes que se env\'ian a un objeto sin mas informaci\'on. Por ejemplo en \ct{3 factorial}, \ct{factorial} es un mensaje unario. 
\item Los \emph{Mensajes Binarios} son mensajes que consisten de operadores (frecuentemente aritm\'eticos). Son binarios porque siempre involucran s\'olo dos objetos: el receptor y el objeto argumento. Por ejemplo en \ct{10 + 20}, \ct{+} es un mensaje binario enviado al receptor \ct{10} con el argumento \ct{20}. 
\item Los \emph{Mensajes Keyword} son mensajes que consisten de una o mas palabras clave, cada una terminada por dos puntos (\ct{:}) y tomando un argumento. Por ejemplo en \ct{anArray at: 1 put: 10}, la palabra clave \ct{at:} toma el argumento \ct{1} y la palabra clave \ct{put:} toma el argumento \ct{10}.
\end{itemize}

%-------------------------------------------------------------
\subsection{Mensajes Unarios}
Los mensajes unarios son mensajes que no requieren ning\'un argumento. Ellos siguen el modelo sint\'actico: \ct{receptor nombreDeMensaje}. El selector se construye simplemente con una sucesi\'on de caracteres que no contengan dos puntos (\ct{:}) (\eg \ct{factorial}, \ct{open}, \ct{class}).
\needlines{4}
\begin{code}{@TEST}
89 sin           --> 0.860069405812453
3 sqrt           --> 1.732050807568877
Float pi         --> 3.141592653589793
'blop' size     --> 4
true not        --> false
Object class --> Object class  "The class of Object is Object class (BANG)"
\end{code}
% ON: I changed the examples to things we can test

\important{Los mensajes unarios son mensajes que no requieren argumentos. Siguen el modelo sint\'actico:\\
\lct{receptor \textbf{selector}}}

%-------------------------------------------------------------
\subsection{Mensajes Binarios} 
Los mensajes binarios son mensajes que requieren exactamente un argumento \emph{y} cuyo selector consiste en una sequencia de uno o mas caracteres del conjunto: \ct{+}, \ct{-}, \ct{*}, \ct{/}, \ct{&}, \ct{=}, \ct{>}, \ct{|}, \ct{<}, \ct{~}, y \ct{@}. Nota que \ct{--} no est\'a permitido por razones de an\'alisis sint\'actico.

\begin{code}{@TEST}
100@100      --> 100@100  "crea un objeto Point"
3 + 4              --> 7
10 - 1            --> 9
4 <= 3            --> false
(4/3) * 3 = 4   --> true  "la igualdad es un simple mensaje binario, las fracciones son exactas"
(3/4) == (3/4) --> false  "dos fracciones iguales no son el mismo objeto"
\end{code}

\important{Los mensajes binarios son mensajes que requieren exactamente un argumento \emph{y} cuyo selector est\'a compuesto por una secuencia de los caracteres \ct{+}, \ct{-}, \ct{*}, \ct{/}, \ct{\&}, \ct{=}, \ct{\>}, \ct{|}, \ct{\<}, \ct{\~}, y \ct{@}. \ct{--} no est\'a permitido. Siguen el modelo sint\'actico:\\
\lct{receptor \textbf{selector} argumento}}

%-------------------------------------------------------------
\subsection{Mensajes Keyword} 

Los mensajes keyword son mensajes que requieren uno o m\'as argumentos y cuyo selector consiste en una o mas palabras clave cada una terminada en dos puntos (\ct{:}). Los mensajes keyword siguen el modelo sint\'actico:  \lct{receptor \textbf{selectorPalabraUno:} argUno \textbf{palabraDos:} argDos}.

Cada palabra clave toma un argumento. Entonces \ct{r:g:b:} es un m\'etodo con tres argumentos, \ct{playFileNamed:} y \ct{at:} son m\'etodos con un argumento, y \ct{at:put:} es un m\'etodo con dos argumentos. Para crear una instancia de la clase \ct{Color} se puede utilizar el m\'etodo \ct{r:g:b:} como en \ct{Color r: 1 g: 0 b: 0}, que crea el color rojo. Nota que los dos puntos son parte del selector.

\important{En \ind{Java} o \ind{C++}, la invocaci\'on al m\'etodo de \st \ct{Color r: 1 g: 0 b: 0}
 ser\'ia escrita \ct{Color.rgb(1, 0, 0)}.}

\begin{code}{@TEST | nums |}
1 to: 10                        --> (1 to: 10)  "crea un intervalo"
Color r: 1 g: 0 b: 0       --> Color red  "crea un nuevo color"
12 between: 8 and: 15 --> true

nums := Array newFrom: (1 to: 5).
nums at: 1 put: 6.
nums --> #(6 2 3 4 5)
\end{code}
% ON: Changed to real examples that we can test

\important{Los mensajes keyword son mensajes que requieren uno o m\'as argumentos. Su selector consiste en una o m\'as palabras clave cada una terminada con dos puntos (\lct{:}). Siguen el modelo sint\'actico:\\
\lct{receptor \textbf{selectorPalabraUno:} argUno \textbf{palabraDos:} argDos}}

%=============================================================
\section{Composici\'on de mensajes}
Los tres tipos de mensajes tienen diferente precedencia, lo cual permite que sean compuestos de manera elegante.

\begin{enumerate}
\item Los mensajes unarios son siempre enviados primero, luego los mensajes binarios y finalmente los mensajes keyword. 
\item Los mensajes entre \ind{par\'entesis} son enviados antes que cualquier otro tipo de mensaje.
\item Los mensajes del mismo tipo son evaluados de izquierda a derecha. 
\end{enumerate}
\index{mensaje!orden de evaluaci\'on}

Estas reglas llevan a un orden de lectura muy natural. Ahora bien, si deseas asegurarte que tus mensajes sean enviados en el orden que deseas puedes agregar par\'entesis como se muestra en la \figref{uKeyUn}. En esta figura, el mensaje \ct{yellow} es un mensaje unario y el mensaje \ct{color:} un mensaje keyword, por lo tanto el env\'io de \ct{Color yellow} se realiza primero. De todos modos, dado que los env\'ios de mensajes entre par\'entesis se realizan primero, colocando par\'entesis (innecesarios) alrededor de \ct{Color yellow} simplemente enfatiza la precedencia de dicho mensaje. El resto de la secci\'on ilustrar\'a cada uno de estos puntos.

\begin{figure}[ht]
\ifluluelse
	{\centerline{\includegraphics[width=0.9\textwidth]{uKeyUn}} }
	{\centerline{\includegraphics[width=10cm]{uKeyUn}} }
\caption{Los mensajes unarios se env\'ian primero, as\'i que se env\'ia \ct{Color yellow}. Esto devuelve un objeto color, el cual es pasado como argumento del mensaje \ct{aPen color:}.\figlabel{uKeyUn}}
\end{figure}

%---------------------------------------------------------
\subsection*{Unarios > Binarios > Keyword}
Los mensaje unarios se env\'ian primero, a continuaci\'on los mensajes binarios, y finalmente los mensajes keyword. Tambi\'en se dice que los mensajes unarios tienen mayor prioridad que los otros tipos de mensaje.

\important{\textbf{Regla Uno.} Los mensaje unarios se env\'ian primero, luego los mensajes binarios, y finalmente los mensajes keyword.\\
\centerline{\ct{Unarios > Binarios > Keyword}}
}

Como muestran estos ejemplos, las reglas de sintaxis de \st generalmente aseguran que los env\'ios de mensajes puedan ser le\'idos de manera natural:
\begin{code}{@TEST}
1000 factorial / 999 factorial --> 1000
2 raisedTo: 1 + 3 factorial     --> 128
\end{code}
\noindent

Desafortunadamente estas reglas son algo simplistas para env\'ios de mensajes aritm\'eticos, entonces es necesario introducir par\'entesis cuando se desea imponer cierta prioridad sobre los operadores binarios:
\begin{code}{@TEST}
1 + 2 * 3   --> 9
1 + (2 * 3) --> 7
\end{code}

El siguiente ejemplo, el cual es un poco mas complejo (!), ofrece una buena ilustraci\'on de que a\'un las expresiones complicadas en \st pueden ser le\'idas de manera natural:
\begin{code}{@TEST}
[:aClass | aClass methodDict keys select: [:aMethod | (aClass >> aMethod) isAbstract ]] value: Boolean --> an IdentitySet(#or: #| #and: #& #ifTrue: #ifTrue:ifFalse: #ifFalse: #not #ifFalse:ifTrue:)
\end{code}
\noindent
Aqu\'i queremos saber cu\'ales m\'etodos de la clase \ct{Boolean} son abstractos\footnote{De hecho, tambi\'en podr\'iamos haber escrito la expresi\'on equivalente pero m\'as simple: \ct{Boolean methodDict select: #isAbstract thenCollect: #selector}}\damien{I've added this footnote, just remove it if you don't like it :-)}.
Le pedimos a alguna clase pasada como argumento, \ct{aClass}, las claves de su diccionario de m\'etodos y seleccionamos aquellas claves asociadas a m\'etodos que son abstractos, utilizando dichas claves para solicitar a la clase el objeto m\'etodo correspondiente y preguntarle a \'este si es abstracto o no.
A continuaci\'on ligamos al argumento \ct{aClass} el valor concreto \ct{Boolean}.
Solamente necesitamos par\'entesis para enviar el mensaje binario \ct{>>}, el cual solicita un m\'etodo a una clase, antes de enviar el mensaje unario \mbox{\ct{isAbstract}} a dicho m\'etodo. El resultado nos muestra qu\'e m\'etodos deben ser implementados por las subclases concretas de \ct{Boolean}: \ct{True} y \ct{False}.

%\begin{code}{}
%Pen new go: 30 + 50          "create a turtle and moves it forward 80 pixels"
%Display restoreAfter: [WarpBlt test4] 					
%	"Keyword message, try test1, test12, test3, test4 and test 5"
%#($t $e $s $t) at: 3 --> $s 
%#($a $b $c $d) at: 2 factorial put: $z 
%\end{code}

%As you can see the syntax and in particular the keyword messages as in
%the example \ct{array at: 1 put: 4} make it possible to write code
%with a structure approaching that of natural language.
% This was one of the initial objectives so that the children can program.

\paragraph{Ejemplo.}
En el mensaje \ct{aPen color: Color yellow}, encontramos el mensaje \emph{unario} \ct{yellow} enviado a la clase \ct{Color} y el mensaje \emph{keyword} \ct{color:} enviado a \ct{aPen}. Los mensajes unarios se env\'ian primero, en consecuencia \ct{Color yellow} es enviado al comienzo de la evaluaci\'on (1). El env\'io realizado devuelve un objeto color, el cual es pasado como argumento del mensaje \ct{aPen color: aColor} (2) como se muestra en el \egref{decColor}.
La \figref{uKeyUn} muestra graficamente c\'omo son enviados los mensajes.

\needlines{5}
\begin{example}[decColor]{Descomponiendo la evaluaci\'on de \ct{aPen color: Color yellow}}{}
        aPen color: Color yellow
(1)                       Color yellow        "mensaje unario enviado primero"
                        --> aColor
(2)   aPen color: aColor                 "mensaje keyword enviado luego"
\end{example}

\paragraph{Ejemplo.} En el mensaje \ct{aPen go: 100 + 20}, encontramos el mensaje \emph{binary} \ct{+ 20} y el mensaje \emph{keyword} \ct{go:}. Los mensajes binarios se env\'ian antes que los mensajes keyword as\'i que \ct{100 + 20} es enviado primero (1): el mensaje \ct{+ 20} es enviado al objeto \ct{100} retornando el número \ct{120}. Entonces el mensaje \ct{aPen go: 120} es enviado con \ct{120} como argumento (2).
El \egref{decGo} muestra como es ejecutado el todo el env\'io de mensajes.

\begin{example}[decGo]{Descomponiendo \ct{aPen go: 100 + 20}}{}
      aPen go: 100 + 20   
(1)                 100 + 20           "mensaje binario enviado primero"
                   -->   120
(2)  aPen go: 120                   "luego el mensaje keyword"
\end{example}

\begin{figure}[htb]
\begin{minipage}{0.48\textwidth}
	\ifluluelse
		{\centerline{\includegraphics[width=0.9\textwidth]{uKeyBin}}}
		{\centerline{\includegraphics[width=6cm]{uKeyBin}}}
	\caption{Los mensajes binarios se env\'ian antes que los mensajes keyword.\figlabel{uKeyBin}}
\end{minipage}
\hfill
\begin{minipage}{0.48\textwidth}
	\begin{center}
	\ifluluelse
		{\includegraphics[width=0.9\textwidth]{uunKeyBin}}
		{\includegraphics[width=6cm]{uunKeyBin}}
\caption{Descomponiendo \ct{Pen new go: 100 + 20}}\figlabel{unKeyBin}
\end{center}
\end{minipage}
\end{figure}

%\begin{figure}[ht]
%\centerline{\includegraphics[width=6cm]{uKeyBin}} 
%\caption{Unary messages are sent first so \ct{Color yellow} is sent. This returns a color object which is passed as argument of the message \ct{aPen color:}.\figlabel{uKeyBin}}
%\end{figure}

%\paragraph{Example 3.}
%The message \ct{aPen penSize: aPen penSize + 2} contains one unary message \ct{penSize}, one binary message \ct{+},  and one keyword message \ct{penSize:}.
%The unary message \ct{aPen penSize} is sent first (1), this message returns a number representing the current size of the receiver pen. Then the binary message is sent (2), the returned number is sent the message \ct{+ 2} which in its turn returns another number. Finally the keyword message 
%\ct{penSize:} is sent with the last number as argument. The expression increases the receiver pen size by two pixels. \Egref{decpen} illustrates the decomposition of message evaluation.

% penSize does not exist on Pen
%\

%\begin{scriptfigwithsize}[0.65]{\includegraphics[width=6cm]{uKeyUnBin}}{Decomposing \ct{ aPen penSize: aPen penSize + 2}}\scrlabel{decpen}
%      aPen penSize: aPen penSize + 2
%(1)                        aPen penSize            "unary"
%                              $\arrow$  aNumber
%(2)                              aNumber + 2	             "binary"
%                                    $\arrow$   anotherNumber	
%(3)   aPen penSize: anotherNumber           "keyword"      
%\end{scriptfigwithsize}

\paragraph{Ejemplo.} Como ejercicio te proponemos descomponer la evaluaci\'on del mensaje \ct{Pen new go: 100 + 20} el cual est\'a compuesto por un mensaje unario, un mensaje keyword y un mensaje binario (ver \figref{unKeyBin}).
% The unary message \ct{Pen new} is first sent. It returns a new bot, then the binary message \ct{100 + 20} is sent and returns \ct{120}. Finally the message \ct{go:} is sent to the newly created robot with \ct{120}.

%\begin{figure}[ht]
%\begin{center}
%\includegraphics[width=8cm]{uunKeyBin}
%\caption{Decomposing \ct{Pen new go: 100 + 20}}\figlabel{unKeyBin}
%\end{center}
%\end{figure}

%-------------------------------------------------------------
\subsection{Los par\'entesis primero}

\important{\textbf{Regla Dos.} Los mensajes entre par\'entesis se env\'ian antes que los otros mensajes.\\
\centerline{\ct{(Mensaje) > Unarios > Binarios > Keyword}}}

\begin{code}{@TEST}
1.5 tan rounded asString = (((1.5 tan) rounded) asString) --> true    "no se necesitan par\'entesis"
3 + 4 factorial   --> 27    "(no es 5040)"
(3 + 4) factorial --> 5040
\end{code}

Aquí necesitamos los \ind{paréntesis} para forzar el envío de \ct{lowMajorScaleOn:} antes del envío de \ct{play}.
\begin{code}{}
(FMSound lowMajorScaleOn: FMSound clarinet) play 
"(1) enviar el mensaje clarinet a la clase FMSound para crear un sonido de clarinete.
 (2) enviar este sonido a FMSound como argumento del mensaje keyword lowMajorScaleOn:.
 (3) reproducir el sonido resultante."
\end{code}

% ON: This has nothing to do with parentheses!
%RecordingControlsMorph new openInWorld
%"An instance of the digitizer is created then visualized. If your microphone is plugged in try a sampleBANG"

% ON: This link is broken, and the result does not understand display!
%(HTTPSocket httpShowGif:
%   'www.altavista.digital.com/av/pix/default/av-adv.gif') display

\paragraph{Ejemplo.}
El mensaje \ct{(65@325 extent: 134 @ 100) center} retorna el centro de un rectángulo cuya esquina superior izquierda es $(65, 325)$ y cuyo tamaño es $134{\times}100$. El \egref{decExtent} muestra cómo el mensaje es descompuesto y enviado. Primero es enviado el mensaje entre paréntesis: contiene los mensajes binarios \ct{65@325} y \ct{134@100} que son enviados primero y retornan objetos punto, y el mensaje keyword \ct{extent:} que es enviado a continuación y retorna un rectángulo. Finalmente es enviado al rectángulo el mensaje unario \ct{center} y entonces un nuevo punto es retornado por este. 
Evaluar el mensaje sin paréntesis llevaría a un error porque el objeto \ct{100} no comprende el mensaje \ct{center}.

\needlines{9}
\begin{example}[decExtent]{Ejemplo de Paréntesis.}{}
      (65 @ 325 extent: 134 @ 100) center
(1)   65@325                                  "binario"
      --> aPoint
(2)   134@100                                 "binario"
      --> anotherPoint
(3)   aPoint extent: anotherPoint      "keyword"
      --> aRectangle
(4)   aRectangle center                    "unario"
      --> 132@375
\end{example}

\subsection{De izquierda a derecha}
Ahora sabemos cómo son manejados los mensajes de distintos tipos o prioridades. La pregunta final que resta por ser respondida es cómo son enviados los mensajes con la misma prioridad. Son enviados de izquierda a derecha. Debes notar que ya viste este comportamiento en \egref{decExtent} donde los dos mensajes (\ct{@}) de creación de puntos fueron enviados primero.

\important{{\textbf{Regla Tres.} Cuando los mensajes son del mismo tipo, el orden de evaluación es de izquierda a derecha.}}

%\begin{figure}
%\centerline{\includegraphics[width=8cm]{ucompoUn}} 
%\caption{The message \ct{Pen new east} is composed of two unary messages. Therefore the leftmost one, \ct{new},  is sent and it returns a new robot to which the second message \ct{east} is sent. \figlabel{compoUn}}
%\end{figure}

\paragraph{Ejemplo.} En los envíos de los mensajes \ct{Pen new down} todos los mensajes son unarios, así que el primero de la izquierda, \ct{Pen new}, se envía primero. Este mensaje retorna un nuevo objeto lápiz al cual se le envía el segundo mensaje \ct{down}, como se muestra en \figref{unaryMessages}.

\begin{figure}
	\centering
	\includegraphics[width=8cm]{ucompoUn}
	\caption{Decomposing \ct{Pen new down}\figlabel{unaryMessages}}
\end{figure}

%-------------------------------------------------------------
\subsection{Inconsistencias aritméticas}
Las reglas de composición de mensajes son simples pero producen inconsistencias en la ejecución de envíos de mensajes aritméticos expresados en mensajes binarios. Aquí veremos algunas situaciones comunes en las cuales el agregado de paréntesis es necesario para subsanar dichas inconsistencias.

\needlines{6}
\begin{code}{@TEST}
3 + 4 * 5      --> 35    "(no resulta 23)  Los mensajes binarios se envían de izquierda a derecha"
3 + (4 * 5)    --> 23
1 + 1/3         --> (2/3)    "no resulta 4/3"
1 + (1/3)       --> (4/3)
1/3 + 2/3       --> (7/9)    "no resulta 1"
(1/3) + (2/3)  --> 1
\end{code}

\paragraph{Ejemplo.} 
En los envíos de mensajes \ct{20 + 2 * 5}, están presenta solamente los mensajes binarios \ct{+} y \ct{*}. En \st no hay prioridades específicas para las operaciones \ct{+} y \ct{*}. Son sólo mensajes binarios, entonces \ct{*} no tiene prioridad por sobre \ct{+}. Aquí el mesaje mas a la izquierda \ct{+} es enviado primero (1) y luego se envía al resultado el mensaje \ct{*} como se muestra en \egref{binaryMessages1}.

\begin{example}[binaryMessages1]{Descomponiendo \ct{20 + 2 * 5}}{}
"Debido a que no hay prioridades entre los mensajes binarios, el mensaje de la izquierda + es evaluado primero, aún cuando para las reglas de la aritmética el * debería ser enviado antes."

      20 + 2 * 5 
(1)  20 + 2 --> 22
(2)  22       * 5 --> 110
\end{example}

\begin{figure}
\begin{center}\includegraphics[width=8cm]{ucompoNoBracketPar}\end{center}
\end{figure}
\noindent
Como se muestra en \egref{binaryMessages1} el resultado de este envío de mensajes no es \ct{30} sino \ct{110}. Este resultado es quizás inesperado pero es directamente consistente con las reglas utilizadas para enviar mensajes. De alguna manera este es un precio a pagar por la simplicidad de modelo de \st. Para obtener el resultado correcto, deberíamos utilizar paréntesis. Los mensajes se evalúan primero cuando son rodeados con paréntesis. Entonces el envío de los mensajes \ct{20 + (2 * 5)} retornará el resultado correcto como se muestra en \egref{mathcorrect}.

\needlines{4}
\begin{example}[mathcorrect]{Descomponiendo \ct{20 + (2 * 5)}}{}
"Los mensajes rodeados con paréntesis son evaluados primero, en consecuencia * se envía antes que + lo cual produce el comportamiento correcto."

    20 + (2 * 5) 
(1)        (2 * 5) --> 10
(2) 20 + 10      --> 30
\end{example}

\begin{figure}
\begin{center}
\includegraphics[width=8cm]{ucompoNumberBracket}
\end{center}
\end{figure}

\important{En \st, los operadores aritméticos como + y * no poseen diferente prioridad. \ct{+} y \ct{*} son simples mensajes binarios, de lo cual se sigue que \ct{*} no posee prioridad por sobre \ct{+}. Se deben utilizar los paréntesis para obtener los resultados deseados.}

%  At the beginning put parenthesis when you have multiple binary messages.}  HUH?  At the beginning of what?!

\begin{figure}
\begin{center}
\ifluluelse
	{\includegraphics[width=\textwidth]{uKeyUnBinPar}}
	{\includegraphics[width=0.8\textwidth]{uKeyUnBinPar}}
\ifluluelse
	{\includegraphics[width=\textwidth]{uunKeyBinPar}}
	{\includegraphics[width=10cm]{uunKeyBinPar}}
\end{center}
\caption{Mensajes equivalentes utilizando paréntesis. \figlabel{uKeyUnBinPar}}
\end{figure}

Se debe notar que la primera regla que determina que los mensajes unarios sean enviados antes que los binarios y que los keyword evita la necesidad de utilizar paréntesis explícitos a su alrededor. La \tabref{expressions} muestra envíos de mensajes escritos siguiendo estas reglas y a su lado los envíos de mensajes equivalentes si las reglas no existieran. Ambos envíos de mensajes resultan en el mismo efecto o retornan el mismo valor.

\begin{figure}\centering
	\begin{tabular}{l@{\qquad}l}
	\toprule
	Precedencias implícita y su equivalentes con paréntesis explícitos\\
	\midrule
	\lct{aPen color: Color yellow}
		& \lct{aPen color: (Color yellow)}
		\\
	\lct{aPen go: 100 + 20}
		& \lct{aPen go: (100 + 20)}
		\\
	\lct{aPen penSize: aPen penSize + 2}
		& \lct{aPen penSize: ((aPen penSize) + 2)}
		\\
	\lct{2 factorial + 4}
		& \lct{(2 factorial) + 4}
		\\
	\bottomrule
	\end{tabular}
	\caption{Envíos de mensajes y equivalencias usando paréntesis.\tablabel{expressions}}
\end{figure}

%=============================================================
\section{Hints for identifying keyword messages}
Frecuentemente los principiantes tienen problemas para comprender cuándo es necesario utilizar paréntesis. Veamos cómo son reconocidos los mensajes keyword por el compilador.

%-------------------------------------------------------------
\subsection{Parentheses or not?}
The characters \ct{[}, \ct{]}, \ct{(} and \ct{)} 
delimit distinct areas. Within such an area, a keyword message is the longest sequence of words terminated  by \ct{:} that is not cut by the characters \ct{.},  or \ct{;}. 
When the characters \ct{[}, \ct{]}, \ct{(} and \ct{)} surround some words with colons, these words participate in the keyword message \emph{local} to the area defined.

In this example, there are two distinct keyword messages:  \ct{rotatedBy:magnify:smoothing:} and \ct{at:put:}.

\begin{code}{}
aDict
   at: (rotatingForm 
          rotateBy: angle	
          magnify: 2 
          smoothing: 1)
   put: 3
\end{code}

\important{
The characters \lct{[}, \lct{]}, \lct{(} and \lct{)} delimit distinct areas. Within such an area, 
a keyword message is the longest sequence  of words terminated by \lct{:} that is not cut by the characters \lct{.},  or \lct{;}. 
When the characters \lct{[}, \lct{]}, \lct{(} and \lct{)} surround some words with colons, these words participate in the keyword message local to the area defined.}

\on{Sounds terribly complicated.}

\paragraph{Hints.} If you have problems with these precedence rules, you may start simply by putting parentheses whenever you want to distinguish two messages having the same precedence.

The following piece of code does not require parentheses because the message send  \ct{x isNil} is unary hence is sent prior to the keyword message \ct{ifTrue:}.
\begin{code}{}
(x isNil)
   ifTrue:[...]
\end{code}

The following piece of code requires parentheses because the messages \ct{includes:} and \ct{ifTrue:} are both keyword messages. 
\begin{code}{}
ord := OrderedCollection new.
(ord includes: $a)
   ifTrue:[...]
\end{code}
\noindent
Without parentheses the unknown message \ct{includes:ifTrue:} would be sent to the collection!

%-------------------------------------------------------------
\subsection{When to use \lct{[ ]} or \lct{( )}}
You may also have problems understanding when to use square brackets rather than parentheses.
The basic principle is that you should use \ct{[ ]} when you do not know how many times, potentially zero, an expression should be evaluated.
\lct{[\emph{expression}]} will create a \ind{block} closure (\ie an object) from \lct{\emph{expression}}, which may be evaluated any number of times (possibly zero), depending on the context. Here note that an expression can either be a message send, a variable, a literal, an assignment or a block.

Hence the conditional branches of \ct{ifTrue:} or \ct{ifTrue:ifFalse:} require blocks. Following the same principle both the receiver and the argument of a \ct{whileTrue:} message require the use of square brackets since we do not know how many times either the receiver or the argument should be evaluated.

Parentheses, on the other hand, only affect the order of sending messages.
So in \lct{(\emph{expression})}, the \lct{\emph{expression}} will \emph{always} be evaluated exactly once.

\begin{code}{}
[ x isReady ] whileTrue: [ y doSomething ]   "both the receiver and the argument must be blocks"
4 timesRepeat: [ Beeper beep ]                   "the argument is evaluated more than once, so must be a block"
(x isReady) ifTrue: [ y doSomething ]           "receiver is evaluated once, so is not a block"
\end{code}

%=============================================================
\section{Expression sequences}
Expressions (\ie messages sends, assignments\dots) separated by periods are evaluated in sequence.
Note that there is no period between a variable definition and the following expression.
The value of a sequence is the value of the last expression. The values returned by all the 
expressions except the last one are ignored. Note that the period is a \subind{statement}{separator} and not a terminator. Therefore a final period is optional.

\begin{code}{@TEST}
| box |
box := 20@30 corner: 60@90.
box containsPoint: 40@50 --> true
\end{code}

%=============================================================
\section{Cascaded messages}
\st offers a way to send multiple messages to the same receiver using a semicolon (\ct{;}). This is called the \emphind{cascade} in \st jargon.

\important{Expression Msg1 ; Msg2}

\begin{minipage}{0.35\textwidth}
\begin{code}{}
Transcript show: 'Pharo is '.
Transcript show: 'fun '.
Transcript cr.
\end{code}
\end{minipage}
\emph{is equivalent to:}
\begin{minipage}{0.35\textwidth}
\begin{code}{}
Transcript        
   show: 'Pharo is';
   show: 'fun ';
   cr
\end{code}
\end{minipage}

Note that the object receiving the cascaded messages can itself be the result of a message send. 
In fact the receiver of all the cascaded messages is the receiver of the first message involved in a cascade. In the following example, the first cascaded message is \ct{setX:setY} since it is followed by a cascade. The receiver of the cascaded message \ct{setX:setY:} is the newly created point resulting from the evaluation of \ct{Point new}, and \emph{not} \ct{Point}. The subsequent message \ct{isZero} is sent to that same receiver. 

\begin{code}{}
Point new setX: 25 setY: 35; isZero --> false
\end{code}

%=============================================================
\section{Chapter summary}

\begin{itemize}
\item A message is always sent to an object named the \emph{receiver} which may be the result of other message sends.

\item Unary messages are messages that do not require any argument.\\
They are of the form of \lct{receiver \textbf{selector}}.

\item Binary messages are messages that involve two objects, the receiver and another object \emph{and} whose selector is composed of  one or more characters from the following list: \ct{+}, \ct{-}, \ct{*}, \ct{/}, \ct{|}, \texttt{\&}, \ct{=}, \ct{>}, \ct{<}, \texttt{\~}, and \ct{@}.
They are of the form: \lct{receiver \textbf{selector} argument}
\item Keyword messages are messages that involve more than one object and that contain at least one colon character (\ct{:}). \\
They are of the form:
\lct{receiver \textbf{selectorWordOne:} argumentOne \textbf{wordTwo:} argumentTwo}

\item \textbf{Rule One.} Unary messages are sent first, then binary messages, and finally keyword messages.
\item \textbf{Rule Two.} Messages in parentheses are sent before any others.
\item \textbf{Rule Three.} When the messages are of the same kind, the order of evaluation is from left to right.
\item In \st, traditional arithmetic operators such as + and * have the same priority. \ct{+} and \ct{*} are just binary messages, therefore \ct{*} does not have priority over \ct{+}. You must use parentheses to obtain a different result.
\end{itemize}

%\end{document}
% ON: Don't ever put an \end{document} in a chapter
% It will make the book stop there!
%=================================================================
\ifx\wholebook\relax\else\end{document}\fi
%=================================================================

%=================================================================
%%% Local Variables:
%%% coding: utf-8
%%% mode: latex
%%% TeX-master: t
%%% TeX-PDF-mode: t
%%% ispell-local-dictionary: "english"
%%% End:

%---------------------------------------------------------
