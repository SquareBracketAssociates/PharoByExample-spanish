% $Author: jdelgado $
% $Date: 2009-12-23 04:18:13 +0100 (Wed, 23 Dec 2009) $
% $Revision: 30033 $

% HISTORY:
% 2006-10-24 - Stef started
% 2006-11-16 - Stef completes first draft
% 2007-04-09 - Andrew review and edit
% 2007-08-23 - Oscar edit
% 2007-09-05 - Andrew edit
% 2010-04-01 - Jordi starts spanish translation
% 2011-06-02 - Jordi finishes first draft
%=================================================================
\ifx\wholebook\relax\else
% -----------------------------------------------
% Lulu:
	\documentclass[a4paper,10pt,twoside]{book}
	\usepackage[
		papersize={6.13in,9.21in},
		hmargin={.75in,.75in},
		vmargin={.75in,1in},
		ignoreheadfoot
	]{geometry}
	\input{../common.tex}
	\pagestyle{headings}
	\setboolean{lulu}{true}
% --------------------------------------------
% A4:
%	\documentclass[a4paper,11pt,twoside]{book}
%	\input{../common.tex}
%	\usepackage{a4wide}
% --------------------------------------------
    \graphicspath{{figures/} {../figures/}}
	\begin{document}
	\renewcommand{\nnbb}[2]{} % Disable editorial comments
	\sloppy
\fi
%=================================================================
\chapter{Clases y metaclases}
\chalabel{metaclasses}

Tal como vimos en el \charef{model}, en \st todo es un objeto, y cada objeto es instancia de alguna clase. Las clases no son ninguna excepci\'on:
las clases son objetos, y dichos objetos son instancias de otras clases.
Este modelo captura la esencia de la programaci\'on orientada a objetos: Es ligero, sencillo, elegante y uniforme.
Sin embargo, las consecuencias de esta uniformidad pueden llegar a confundir a los ne\'ofitos. El objetivo de este cap\'{\i}tulo es mostrar que no es complicado, ``m\'agico'' o especial: tan s\'olo son reglas sencillas aplicadas de manera uniforme.
Siguiendo estas reglas uno siempre puede entender por qu\'e el sistema se comporta de la manera observada.


%=================================================================
\section{Reglas para las clases y las metaclases}

El modelo de objetos de  \st est\'a basado en un n\'umero limitado de conceptos aplicados de manera uniforme. Los dise\~nadores de Smalltalk aplicaron la navaja de Occam: descartar cualquier consideraci\'on que diera lugar a un sistema m\'as complicado.

Recordemos las reglas del modelo de objetos que exploramos en \charef{model}.

\begin{enumerate}[label={\textbf{Regla \arabic{*}}.}, ref={Regla \arabic{*}}, leftmargin=*, widest=10]
% NB: rule labels must not be multiply defined across chapters!
\item{} % \rulelabel{everything}
	Todo es un objeto.

\item{} % \rulelabel{instance}
	Todo objeto es instancia de una clase.

\item{} % \rulelabel{inheritance}
	Toda clase tiene una superclase.

\item{} % \rulelabel{message}
	Todo ocurre mediante el env\'{\i}o de mensajes.

\item{} % \rulelabel{lookup}
	La b\'usqueda de m\'etodos sigue la cadena de herencia.

\end{enumerate}

Tal y como mencionamos en la introducci\'on a este cap\'{\i}tulo, una consecuencia de la \ref{rule:everything} es que \emph{las clases tambi\'en son objetos}, de modo que la \ref{rule:instance} nos dice que las clases deben ser tambi\'en instancias de otras clases. Llamaremos \emph{metaclase} a la clase de una clase.
\seclabel{metaclassIntro}
Una \indmain{metaclase} se crea autom\'aticamente cuando se crea una clase. La mayor\'{\i}a de las veces no hace falta preocuparse de las metaclases. Sin embargo, cada vez que se usa el navegador para explorar el ``\subind{browser}{lado de clase}'' de una clase, es conveniente recordar que en realidad se est� explorando una clase diferente. Una clase y su metaclase son dos clases distintas, aunque la primera sea una instancia de la segunda.

Para poder explicar las clases y las metaclases adecuadamente, necesitamos extender las reglas del \charef{model} con las siguientes reglas adicionales.

\begin{enumerate}[label={\textbf{Rule \arabic{*}}.}, ref={Rule \arabic{*}}, leftmargin=*, widest=10]
\setcounter{enumi}{5}
\item{} \rulelabel{metaclass}
  Toda clase es instancia de una metaclase.

\item{} \rulelabel{parallelhierarchy}
  La \subind{metaclass}{jerarqu\'{\i}a} de metaclases tiene la misma estructura que la jerarqu\'{\i}a de clases.

\item{} \rulelabel{behavior}
  Toda metaclase hereda de \clsind{Class} y de \clsind{Behavior}.

\item{} \rulelabel{metaclassclass}
  Toda metaclase es instancia de \clsind{Metaclass}.

\item{} \rulelabel{metaclassmetaclass}
  La metaclase de \ct{Metaclass} es instancia de \ct{Metaclass}.

\end{enumerate}

\noindent
Estas 10 reglas definen el modelo de objetos de \st.

Primero revisaremos brevemente las 5 reglas del \charef{model} con un ejemplo peque\~no. Despu\'es examinaremos en detalle las nuevas reglas, a partir del mismo ejemplo.

%=================================================================
\section{Repaso del modelo de objetos de Smalltalk}

Debido a que todo es un objeto, el color azul (\emph{blue}) tambi\'en es un objeto.
\begin{code}{@TEST}
Color blue --> Color blue
\end{code}

\noindent
Cada objeto es instancia de una clase. La clase del color azul es la clase  \clsind{Color}:
\begin{code}{@TEST}
Color blue class --> Color
\end{code}

\noindent
Es interesante observar que si asignamos un cierto valor \emph{alpha} a un color, obtenemos una instancia de una clase diferente, a saber \clsind{TranslucentColor}: 
\begin{code}{@TEST}
(Color blue alpha: 0.4) class --> TranslucentColor
\end{code}

\noindent
Podemos crear un morph y asignar este color transl\'ucido como color del morph: 
\begin{code}{}
EllipseMorph new color: (Color blue alpha: 0.4); openInWorld
\end{code}
\noindent
Podemos ver lo que sucede en la  \figref{translucentEllipse}.

\begin{center}
\begin{figure}[!ht]
\ifluluelse
	{\centerline {\includegraphics[scale=0.7]{TranslucentEllipse}}}
	{\centerline {\includegraphics[width=8cm]{TranslucentEllipse}}}
\caption{Una el\'{\i}pse transl\'ucida\figlabel{translucentEllipse}}
\end{figure}
\end{center}

Por la \ref{rule:inheritance}, cada clase tiene una superclase.
La superclase de \clsind{TranslucentColor} es \clsind{Color}, y la superclase de \ct{Color} es \clsind{Object}:
\begin{code}{@TEST}
TranslucentColor superclass --> Color
Color superclass                   --> Object
\end{code}

Todo ocurre mediante el {\index{sending messages}env\'{\i}o de mensajes\xspace} (\ref{rule:message}), de modo que podemos deducir que \mthind{Color class}{blue} es un mensaje para \ct{Color}, \mthind{Object}{class} y \mthind{Color}{alpha:} son mensajes para el color azul (\emph{blue}),  \mthind{Morph}{openInWorld} es un mensaje para un morph elipse, y \mthind{Behavior}{superclass} es un mensaje para \ct{TranslucentColor} y para \ct{Color}. En cada caso el receptor es un objeto, ya que todo es un objeto, pero algunos de estos objetos son tambi\'en clases.

La b\'usqueda de m\'etodo sigue la cadena de herencia (\ref{rule:lookup}), de modo que cuando se env\'{\i}a el mensaje \ct{class} al resultado de \ct{Color blue alpha: 0.4}, el mensaje es tratado cuando se encuentra el correspondiente m\'etodo en la clase \ct{Object}, como podemos ver en la \figref{classmessage}.

\Begin{center}
\begin{figure}[!ht]
\ifluluelse
	{\centerline{\includegraphics[width=\textwidth]{TranslucentClassMessage}}}
	{\centerline{\includegraphics[width=0.8\textwidth]{TranslucentClassMessage}}}
\caption{Env\'{\i}o de un mensaje a un color transl\'ucido\figlabel{classmessage}}
\end{figure}
\end{center}

La figura captura la esencia de la relaci\'on \emphind{es-un}.
Nuestro objeto azul transl\'ucido \emph{es-una} instancia de \ct{TranslucentColor}, pero tambi\'en podemos decir que \emph{es-un} \ct{Color} y que \emph{es-un} \ct{Object}, ya que responde a mensajes definidos en todas estas clases. De hecho, hay un mensaje, \mthind{Object}{isKindOf:}, que podemos enviar a cualquier objeto para saber si est\'a en una relaci\'on \emph{es-un} con una clase determinada:
\begin{code}{@TEST | translucentBlue |}
translucentBlue := Color blue alpha: 0.4.
translucentBlue isKindOf: TranslucentColor --> true
translucentBlue isKindOf: Color                    --> true
translucentBlue isKindOf: Object                  --> true
\end{code}

%=================================================================
\section{Toda clase es instancia de una metaclase}

% \ruleref{metaclass}

Tal como mencionamos en \secref{metaclassIntro}, las clases cuyas instancias son clases son llamadas {\index{metaclass}metaclases\xspace}.

\paragraph{Las metaclases son impl\'{\i}citas.}
Las metaclases se crean autom\'aticamente cuando se define una clase. Decimos que son \emph{impl\'{\i}citas}, ya que el programador nunca tiene por qu\'e preocuparse por ellas. Una metaclase \subind{metaclass}{impl\'{\i}cita} es creada por cada clase que se crea, de modo que cada metaclase tiene una sola instancia.

Mientras que los nombres de las clases normales son variables globales, las metaclases son an\'onimas. Sin embargo siempre podemos referirnos a las metaclases usando las clases que son sus instancias. La clase de \clsind{Color}, por ejemplo, es \clsind{Color class}, y la clase de \ct{Object} es \clsind{Object class}:
\begin{code}{@TEST}
Color class   --> Color class
Object class --> Object class
\end{code}

\noindent
La \figref{translucentmetaclasses} muestra que cada clase es instancia de su metaclase (\subind{metaclass}{an\'onima}). 

\begin{center}
\begin{figure}[!ht]
\ifluluelse
	{\centerline {\includegraphics[width=\textwidth]{TranslucentMetaclasses}}}
	{\centerline {\includegraphics[width=0.8\textwidth]{TranslucentMetaclasses}}}
\caption{Las metaclases de Translucent y sus superclases\figlabel{translucentmetaclasses}}
\end{figure}
\end{center}

El hecho de que las clases sean tambi�n objetos nos facilita la obtenci�n de informaci\'on de dichas clases mediante el env\'{\i}o de mensajes. Veamos:

\begin{code}{@TEST}
Color subclasses                           --> {TranslucentColor}
TranslucentColor subclasses         --> #()
TranslucentColor allSuperclasses  --> an OrderedCollection(Color Object ProtoObject)
TranslucentColor instVarNames     --> #('alpha')
TranslucentColor allInstVarNames --> #('rgb' 'cachedDepth' 'cachedBitPattern' 'alpha')
TranslucentColor selectors             -->  an IdentitySet(#pixelValueForDepth: #pixelWord32 #convertToCurrentVersion:refStream: #isTransparent #scaledPixelValue32 #bitPatternForDepth: #storeArrayValuesOn: #setRgb:alpha: #alpha #isOpaque #pixelWordForDepth: #isTranslucentColor #hash #isTranslucent #alpha: #storeOn: #asNontranslucentColor #privateAlpha #balancedPatternForDepth:)
\end{code}
\cmindex{Class}{subclasses}
\cmindex{Behavior}{allSuperclasses}
\cmindex{Behavior}{instVarNames}
\cmindex{Behavior}{allInstVarNames}
\cmindex{Behavior}{selectors}

%=================================================================
\section{La jerarqu\'{\i}a de metaclases tiene la misma estructura que la jerarqu\'{\i}a de clases}
% \ruleref{parallelhierarchy}

La \ref{rule:parallelhierarchy} nos dice que la superclase de una metaclase no puede ser una clase arbitraria: Est\'a constre\~nida por la metaclase de la superclase de la \'unica instancia de la metaclase.

\begin{code}{@TEST}
TranslucentColor class superclass --> Color class
TranslucentColor superclass class --> Color class
\end{code}

\noindent
Esto es lo que queremos decir cuando decimos que la \subindmain{metaclass}{jerarqu\'{\i}a} de metaclases tiene la misma estructura que la jerarqu\'{\i}a de clases; La \figref{parallelHierarchies} muestra la estructura de la jerarqu\'{\i}a de la clase \clsind{TranslucentColor}.

\begin{center}
\begin{figure}[!ht]
\ifluluelse
	{\centerline {\includegraphics[width=\textwidth]{TranslucentMetaclassHierarchy}}}
	{\centerline {\includegraphics[width=0.8\textwidth]{TranslucentMetaclassHierarchy}}}
\caption{La jerarqu\'{\i}a de metaclases tiene la misma estructura que la jerarqu\'{\i}a de clases.\figlabel{parallelHierarchies}}
\end{figure}
\end{center}

\begin{code}{@TEST}
TranslucentColor class                                     --> TranslucentColor class
TranslucentColor class superclass                   --> Color class
TranslucentColor class superclass superclass --> Object class
\end{code}

\paragraph{Uniformidad entre Clases y Objetos.}
Es interesante tomarse un momento para ver las cosas con cierta perspectiva: No hay ninguna diferencia entre enviar un mensaje a un objeto y a una clase. En ambos casos la b\'usqueda del m\'etodo correspondiente empieza en la clase del receptor, y posteriormente sigue la cadena de herencia.

As� pues, los mensajes enviados a clases deben seguir la cadena de herencia de las metaclases. Consideremos, por ejemplo, el m\'etodo \mthind{Color class}{blue}, que est\'a implementado en el \subind{browser}{lado de clase} de \ct{Color}. Si enviamos el mensaje \ct{blue} a \ct{TranslucentColor}, el m\'etodo correspondiente ser\'a buscado de la misma manera que cualquier otro mensaje. La b\'usqueda empieza en \ct{TranslucentColor class}, y prosigue siguiendo la jerarqu\'{\i}a hasta que dicho m\'etodo es encontrado en \ct{Color class} (ver la \figref{metaclasslookup}).

\begin{code}{@TEST}
TranslucentColor blue --> Color blue
\end{code}
\noindent
Notemos que obtenemos como resultado un \ct{Color blue} normal, y no uno transl\'ucido\,---\, !'sin magia!

\begin{center}
\begin{figure}[!ht]
\ifluluelse
	{\centerline {\includegraphics[width=\textwidth]{TranslucentColorBlue}}}
	{\centerline {\includegraphics[width=0.8\textwidth]{TranslucentColorBlue}}}
\caption{La b\'usqueda de los m\'etodos correspondientes a mensajes enviados a clases es exactamente igual a la que ya conoc\'{\i}amos para los objetos normales\figlabel{metaclasslookup}}
\end{figure}
\end{center}

As\'{\i} pues, vemos que hay un s\'olo tipo uniforme de \subind{method}{b\'usqueda} de m\'etodos en \st. Las clases son objetos, y se comportan como cualquier otro objeto. Las clases tienen la capacidad de crear nuevas instancias tan s\'olo porque las clases pueden responder al mensaje \ct{new}, y porque el m\'etodo para \ct{new} sabe c\'omo crear instancias nuevas. Normalmente, los objetos que no son clases no comprenden este mensaje, pero si existe alguna buena raz\'on para hacerlo no hay nada que impida al programador a\~nadir un m\'etodo \ct{new} a una clase que no sea metaclase.

Como las clases son objetos, podemos inspeccionarlas.
\index{inspector}

\dothis{Inspect \ct{Color blue} and \ct{Color}.}

\noindent
Fij\'emonos que en un caso estamos inspeccionando una instancia de \ct{Color} y en el otro caso la misma clase \ct{Color}. Esto puede ser un poco confuso, porque la barra de t\'{\i}tulo del inspector nombra la \emph{clase} del objeto inspeccionado.

El inspector aplicado a \ct{Color} nos permite ver la superclase, las variables de instancia, el \subind{method}{diccionario} de m\'etodos, etc. de la clase \ct{Color}, como podemos ver en la \figref{inspectingColor}.

\begin{center}
\begin{figure}[!ht]
\ifluluelse
	{\centerline{\includegraphics[width=\textwidth]{InspectingColor}}}
	{\centerline{\includegraphics[width=10cm]{InspectingColor}}}
\caption{Las clases tambi\'en son objetos.\figlabel{inspectingColor}}
\end{figure}
\end{center}

%=================================================================
\section{Toda metaclase hereda de \lct{Class} y de \lct{Behavior}}
% \ruleref{behavior}

Toda \ind{metaclase} \emphind{es-una} clase, por lo tanto hereda de \clsind{Class}.
\ct{Class} a su vez hereda de sus superclases, \clsind{ClassDescription} y \clsind{Behavior}.
Como todo en \st \emph{es-un} objeto, todas estas clases eventualmente heredan de \ct{Object}.
Podemos ver el panorama completo en la \figref{inheritbehavior}.

\begin{center}
\begin{figure}
\ifluluelse
	{\centerline{\includegraphics[width=\textwidth]{TranslucentBehavior}}}
	{\centerline{\includegraphics[width=0.8\textwidth]{TranslucentBehavior}}}
\caption{Las metaclases heredan de \lct{Class} y \lct{Behavior}\figlabel{inheritbehavior}}
\end{figure}
\end{center}

\paragraph{?'D\'onde est\'a definido \lct{new}?}
Para entender la importancia del hecho que las metaclases hereden de \ct{Class} y \ct{Behavior}, ayuda preguntarse d\'onde est\'a definido \ct{new} y c\'omo encontrarlo. Cuando el mensaje \ct{new} es enviado a una clase, este es buscado en su cadena de metaclases y finalmente en sus superclases \ct{Class}, \ct{ClassDescription} y \ct{Behavior} como se puede ver en la \figref{sendingnew}.

La pregunta \emph{?'D\'onde est\'a definido \ct{new}?} es crucial. \mthind{Behavior}{new} se define primero en la clase \ct{Behavior}, y puede ser redefinido en sus subclases, incluyendo cualquiera de las metaclases de las clases que definimos si resulta necesario. Cuando se env\'{\i}a el mensaje \ct{new} a una clase, este se busca, como siempre, en la metaclase de esta clase, subiendo por la cadena de superclases hasta la clase \ct{Behavior}, si no ha sido redefinido por el camino.

El resultado de enviar \ct{TranslucentColor new} es una instancia de \clsind{TranslucentColor} y \emph{no} de \ct{Behavior}, !'aunque el metodo se haya encontrado en la clase \ct{Behavior}! \ct{new} siempre devuelve una instancia de \self, la clase que recibe el mensaje, incluso si est\'a implementado en otra clase.

\begin{code}{@TEST}
TranslucentColor new class --> TranslucentColor    "!'no Behavior!"
\end{code}

\begin{center}
\begin{figure}
\ifluluelse
	{\centerline{\includegraphics[width=\textwidth]{TranslucentSendingNew}}}
	{\centerline{\includegraphics[width=0.8\textwidth]{TranslucentSendingNew}}}
\caption{\lct{new} es un mensaje normal que se busca en la cadena de metaclases.\figlabel{sendingnew}}
\end{figure}
\end{center}

Un error com\'un es buscar \ct{new} en la superclase de la clase receptora del mensaje. Lo mismo sucede para \ct{new:}, el mensaje est\'andar para crear objetos de un tama\~no determinado. Por ejemplo, \lct{Array new: 4} crea un arreglo de 4 elementos. Este m\'etodo no se encuentra definido en \ct{Array} o ninguna de sus superclases. Hay que buscarlo en \ct{Array class} y sus superclases, ya que es ah\'{\i} donde la b\'usqueda va a empezar.

\paragraph{Responsabilidades de \lct{Behavior}, \lct{ClassDescription} y \lct{Class}.}
\clsind{Behavior} proporciona el estado m\'{\i}nimo necesario para objetos que tienen instancias: Esto incluye un enlace a superclase, un diccionario de m\'etodos y una descripci\'on de las instancias (\ie representaci\'on y n\'umero).
\ct{Behavior} hereda de \ct{Object}, as\'{\i} que la \ct{Behavior}, y todas sus subclases, pueden comportarse como objetos.

\ct{Behavior} es tambi\'en la interfaz b\'asica del compilador. Proporciona m\'etodos para crear un diccionario de m\'etodos, compilar m\'etodos, crear instancias (\ie \mthind{Behavior}{new}, \mthind{Behavior}{basicNew}, \mthind{Behavior}{new:}, y \mthind{Behavior}{basicNew:}), gestionar la jerarqu\'{\i}a de clases (\ie \mthindex{Behavior}{superclass:} \lct{superclass:}, \mthind{Behavior}{addSubclass:}),
acceder a m\'etodos (\ie \mthind{Behavior}{selectors}, \lmthind{Behavior}{allSelectors}, \mthind{Behavior}{compiledMethodAt:}),
acceder a instancias y variables (\ie \mthind{Behavior}{allInstances}, \mthind{Behavior}{instVarNames}\ldots),
acceder a la jerarqu\'{\i}a de clases (\ie \mthind{Behavior}{superclass}, \mthind{Behavior}{subclasses})
y obtener informaci\'on diversa (\ie \mthind{Behavior}{hasMethods}, \mthind{Behavior}{includesSelector}, \mthind{Behavior}{canUnderstand:}, \mthind{Behavior}{inheritsFrom:}, \mthind{Behavior}{isVariable}).

\clsind{ClassDescription} es una clase abstracta que proporciona utilidades requeridas por sus dos subclases m\'as directas, \clsind{Class} y \clsind{Metaclass}. \clsind{ClassDescription} a\~nade ciertas utilidades a las utilidades b\'asicas proporcionadas por \ct{Behavior}:
variables de instancia con nombre, categorizaci\'on de m\'etodos en protocolos, noci\'on de nombre (abstracta), mantenimiento de conjuntos de cambios y el registro de los cambios, y la mayor\'{\i}a de los mecanismos para guardar los cambios en ficheros (\emph{filing-out})

\clsind{Class} representa el comportamiento com\'un a todas las clases.
Proporciona un nombre a la clase, m\'etodos de compilaci\'on, almacenamiento de m\'etodos y variables de instancia. Proporciona una representaci\'on concreta para los nombres de las variables de clase y las variables compartidas (\emph{pool})  (\mthind{Class}{addClassVarName:}, \mthind{Class}{addSharedPool:}, \mthind{Class}{initialize}). \ct{Class} sabe c\'omo crear instancias, de modo que todas las metaclases deber\'{\i}an heredar de \ct{Class}.


%=================================================================
\section{Toda metaclase es instancia de \lct{Metaclass}}
% \ruleref{metaclassclass}

Las metaclases tambi\'en son objetos; son instancias de la clase \clsind{Metaclass}, como se muestra en la \figref{metaclassclass}. Las instancias
de la clase \ct{Metaclass} son metaclases an\'onimas, y cada una tiene ex\'actamente una instancia, que es una clase.

\begin{center}
\begin{figure}
\ifluluelse
	{\centerline{\includegraphics[width=\textwidth]{TranslucentMetaclassClass}}}
	{\centerline{\includegraphics[width=0.8\textwidth]{TranslucentMetaclassClass}}}
\caption{Toda metaclase es una \lct{Metaclass}.\figlabel{metaclassclass}}
\end{figure}
\end{center}

\ct{Metaclass} representa el comportamiento com\'un a las metclases. Proporciona m\'etodos para la creaci\'on de instancias  (\lct{sub\-class\-Of:}), para crear instancias inicializadas de la \'unica instancia de la metclase, inicializaci\'on de variables de clase, para la instancia de la metaclase, compilaci\'on de m\'etodos e informaci\'on de clase (enlaces de herencia, variables de instancia, \etc).

%=================================================================
\section{La metaclase de \lct{Metaclass} es instancia de \lct{Metaclass}}
% \ruleref{metaclassmetaclass}

La pregunta final que queda por responder es: ?'Cual es la clase de \clsind{Metaclass class}?

La respuesta es sencilla: Es una metaclase, de modo que debe ser una instancia de \ct{Metaclass}, como el resto de metaclases del sistema (ver la \figref{metaclassclassclass}).

\begin{center}
\begin{figure}
\ifluluelse
	{\centerline{\includegraphics[width=\textwidth]{TranslucentMetaclassClassClass}}}
	{\centerline{\includegraphics[width=0.8\textwidth]{TranslucentMetaclassClassClass}}}
\caption{Todas las metaclases son instancias de la clase \lct{Metaclass}, incluso la metaclase de \lct{Metaclass}.\figlabel{metaclassclassclass}}
\end{figure}
\end{center}

La figura muestra como todas las metaclases son instancias de \ct{Metaclass}, incluso la metaclase de \ct{Metaclass}. Si comparamos las figuras \ref{fig:metaclassclass} y \ref{fig:metaclassclassclass} veremos como la \subind{metaclass}{jerarqu\'{\i}a} de metaclases perfectamente refleja la estructura de la jerarqu\'{\i}a de clases, ascendiendo hasta \ct{Object class}.

Los siguientes ejemplos muestran como podemos obtener informaci�n de la jerarqu\'{\i}a de clases para demostrar que la \figref{metaclassclassclass} es correcta (de hecho, veremos que hemos contado una peque\~na mentira\,---\,\ct{Object class superclass -->} \clsind{ProtoObject class}, no \ct{Class}. En \pharo debemos subir una superclase m�s para llegar a \ct{Class}.)

\begin{example}{La jerarqu\'{\i}a de clases}{@TEST}
TranslucentColor superclass --> Color
Color superclass                   --> Object
\end{example}

\needlines{4}
\begin{example}{La jerarqu\'{\i}a paralela de metaclases}{@TEST}
TranslucentColor class superclass   --> Color class
Color class superclass                     --> Object class
Object class superclass superclass --> Class    "NB: skip ProtoObject class"
Class superclass                              --> ClassDescription
ClassDescription superclass            --> Behavior
Behavior superclass                         --> Object
\end{example}

\begin{example}{Instancias de Metaclass}{@TEST}
TranslucentColor class class --> Metaclass
Color class class                   --> Metaclass
Object class class                 --> Metaclass
Behavior class class              --> Metaclass
\end{example}
\begin{example}{Metaclass class es una Metaclass}{@TEST}
Metaclass class class --> Metaclass
Metaclass superclass --> ClassDescription
\end{example}

%=================================================================
\section{Resumen del cap\'{\i}tulo}

El lector deber\'{\i}a entender mejor como estan organizadas las clases y la relevancia de un modelo de objetos uniforme. Si resulta confuso siempre hay que recordar que la clave est\'a en el env\'{\i}o de mensajes: Hay que buscar el m\'etodo en la clase del receptor. Esto es as\'{\i} para \emph{cualquier} receptor. Si el m\'etodo no se encuentra en la clase del receptor, se busca en sus superclases.

\begin{itemize}
\item Toda clase es instancia de una metaclase.
	Las metaclases son impl\'{\i}citas. Una metaclase se crea autom\'aticamente cuando se crea la clase que es su \'unica instancia.

\item La \subind{metaclass}{jerarqu\'{\i}a} de metaclases tiene la misma estructura que la jerarqu\'{\i}a de clases.
	La b\'usqueda de m\'etodos para las clases sigue el mismo procedimiento que sigue en los objetos corrientes, y asciende por la cadena de superclases de la metaclase.

\item Toda metaclase hereda de \clsind{Class} y de \clsind{Behavior}.
        Toda clase \emph{es una} \ct{Class}. Como las metaclases tambi\'en son clases, deben heredar de \ct{Class}. \ct{Behavior} proporciona comportamiento com\'un a todas las entidades capaces de tener instancias.

\item Toda metaclase es instancia de \clsind{Metaclass}.
	\ct{ClassDescription} proporciona todo aquello que es com\'un a \ct{Class} y a \ct{Metaclass}.

\item La metaclase de \ct{Metaclass} es instancia de \ct{Metaclass}.
	La relaci\'on \emph{instancia-de} forma un ciclo, de modo que \ct{Metaclass class class --> Metaclass}.
\end{itemize}
%=================================================================
\ifx\wholebook\relax\else\end{document}\fi
%=================================================================

%-----------------------------------------------------------------
