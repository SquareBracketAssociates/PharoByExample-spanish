% $Author$
% $Date$
% $Revision$
% $Id$

% HISTORY:
% 2006-10-24 - Stef started
% 2006-12-01 - Oscar edit
% 2006-12-02 - Andrew edit
% 2007-05-04 - Oscar first draft
% 2007-07-04 - Stef review
% 2009-12-18 - Rafa starts spanish translation

%=================================================================
\ifx\wholebook\relax\else
% --------------------------------------------
% Lulu:
	\documentclass[a4paper,10pt,twoside]{book}
	\usepackage[
		papersize={6.13in,9.21in},
		hmargin={.75in,.75in},
		vmargin={.75in,1in},
		ignoreheadfoot
	]{geometry}
	\input{../common.tex}
	\pagestyle{headings}
	\setboolean{lulu}{true}
% --------------------------------------------
% A4:
%	\documentclass[a4paper,11pt,twoside]{book}
%	\input{../common.tex}
%	\usepackage{a4wide}
% --------------------------------------------
    \graphicspath{{figures/} {../figures/}}
	\begin{document}
%	\renewcommand{\nnbb}[2]{} % Disable editorial comments
	\sloppy
\fi
\renewcommand\yellowbox[1]{\relax}
\renewcommand\nnbb[2]{\relax} 

%=================================================================
\chapter{Sintaxis en dos palabras}
\luq{\'¿Qu\'e tal 'Sintaxis en resumen' o 'Sintaxis en pocas palabras'?}

\nicopaez{Voto por 'Sintaxis en pocas palabras'}
\chalabel{syntax}

\pharo, como la mayor\'ia de los dialectos de \st modernos, adopta una sintaxis muy similar a la de \st-80.
La \ind{sintaxis} est\'a dise\~nada para que el texto del programa pueda leerse como una especie de ingl\'es macarr\'onico: \luq{\'¿o mejor lengua franca del ingl\'es?} \nicopaez{me gusta mas lengua franca del frances}

\begin{code}{}
(Smalltalk includes: Class) ifTrue: [ Transcript show: Class superclass ]
\end{code}

\noindent
La sintaxis de \pharo es m\'inima.
Esencialmente, s\'olo hay sintaxis para el \emph{env\'io de mensajes} (es decir, expresiones).
Las expresiones se construyen a partir de un n\'umero muy peque\~no de elementos primitivos.
Hay s\'olo 6 palabras clave y no existe sintaxis para las estructuras de control o la declaraci\'on de nuevas clases.
En su lugar, casi todo se consigue enviando mensajes a objetos.
Por ejemplo, en lugar de una estructura de control if-then-else, \st env\'ia mensajes como \ct{ifTrue:} a objetos \clsind{Boolean}.
Las nuevas (sub)clases se crean enviando un mensaje a sus superclases.

%=================================================================
\section{Elementos sint\'acticos}

Las expresiones se componen de las siguientes piezas:
(i) seis palabras reservas \nicopaez{me parece que aqui en lugar de reservas seria reservadas} o \emph{pseudo-variables}: 
\pvind{self}, \pvind{super}, \pvind{nil}, \pvind{true}, \pvind{false} y \pvind{thisContext}; 
(ii) expresiones constantes para \emphind{objetos literales}
incluyendo n\'umeros, caracteres, cadenas de caracteres, s\'imbolos y \emph{arrays}\luq{\'¿traduccimos array?};\nicopaez{podriamos traducirlo como arreglo, no suena tan mal}
(iii) declaraciones de variables; 
(iv) asignaciones; 
(v) \ind{bloques} de cierre \luq{\'¿traducci\'on?} \nicopaez{y si solo decimos bloques}y
(vi) mensajes.

\seeindex{pseudo-variable}{variable, pseudo}

\begin{table}\centering
	\begin{tabular}{ll}
		\toprule
		Sintaxis & Lo que representa \\
		\midrule
		\lct{puntoInicial}			&	un nombre de variable\\
		\lct{Transcript}			&	un nombre de variable global\\
		\lct{self}				&	una pseudo-variable \\
		\midrule
		\lct{1}				 	&	entero decimal \\
		\lct{2r101}				&	entero binario \\
		\lct{1.5}					&	n\'umero en coma flotante \\
        \luq{Usamos coma flotante o punto flotante?} \nicopaez{para mi seria punto flotante}\\
		\lct{2.4e7}				&	notaci\'on exponencial \\
		\lct{\$a}					&	el car\'acter `a' \\
		\lct{'Hola'}				&	la cadena de caracteres ``Hola'' \\
		\lct{\#Hola}				&	el s\'imbolo \lct{\#Hola} \\
		\lct{\#(1 2 3)}			& un array literal \\
		\lct{\{1. 2. 1+2\}}		& un array din\'amico \\
		\midrule
		\lct{"un comentario"} 		&	un comentario \\
		\midrule
		\lct{| x y |}				&	declaraci\'on de las variables \lct{x} e \lct{y}	\\
		\lct{x := 1}				&	asignar 1 a \lct{x} \\
		\lct{[ x + y ]}			&	un bloque que se eval\'ua como \lct{x+y} \\
		\lct{<primitive: 1>}		&	primitiva o anotaci\'on de la m\'aquina virtual\\
		\midrule
		\lct{3 factorial}			&	mensaje unario \\
		\lct{3+4}					&	mensaje binario \\
		\lct{2 raisedTo: 6 modulo: 10}		&	mensaje de palabra clave \\
		\midrule
		\lct{$\uparrow$ true} 			&	devuelve el valor true	\\
		\lct{Transcript show: 'hola'. Transcript cr }		&	separator de expresiones (\lct{.})	\\
		\lct{Transcript show: 'hola'; cr}					&	mensaje en cascada (\lct{;}) \\
		\bottomrule
	\end{tabular}
	\caption{Sintaxis de \pharo en dos palabras\tablabel{syntax}}
\end{table}

Podemos ver ejemplos de los distintos elementos sint\'acticos en la \tabref{syntax}.
\begin{description}
\item[Variables locales] \ct{puntoInicial} es un nombre de variable, o identificador.
  Por convenio,\nicopaez{no seria mejor decir 'por convencion'} los identificadores est\'an formados por palabras en ``\ind{camelCase}'' (es decir, cada palabra excepto la primera comienza con una letra may\'uscula).
  La primera letra de una variable de instancia, m\'etodo, argumento de un bloque o variable temporal debe ser min\'uscula.
  Esto indica al lector que la variable tiene un alcance privado.

\item[Variables compartidas] Los identificadores que comienzan con letras may\'usculas son variables \subind{variable}{globales}, variables \subind{variable}{de clase}, diccionarios comunes (\subind{variable}{pool} dictionaries) o nombres de clase.
\luq{pool dictionaries?}\nicopaez{no se me ocurre una buena traduccion, por eso lo dejaria en ingles}
		\ct{Transcript} es una variable global, una instancia de la clase \ct{TranscriptStream}.
		\seeindex{variable global}{variable, global}
		\seeindex{diccionario com\'un}{variable, pool}
		\seeindex{variable!de clase}{clase, variable}

\item[El receptor] \pvind{self} es una palabra clave que se refiere al objeto en el que el m\'etodo actual se est\'a ejecutando. Lo llamamos ``el receptor'' porque normalmente este objeto habr\'a recibido el mensaje que caus\'o que el m\'etodo se ejecutara.
		Decimos que \self es una ``\subind{variable}{pseudo}-variable'' ya que no podemos asignarla.

\item[Enteros] Adem\'as de los enteros decimales normales como \ct{42}, \pharo tambi\'en proporciona una \ind{notaci\'on cient\'ifica}.
	\ct{2r101} es \ct{101} en base 2 (es decir, binario), que es igual al n\'umero decimal 5.
	\index{literal}
	\index{literal!n\'umero}

\item[N\'umeros en coma flotante] Pueden especificarse en potencias de base diez: \mbox{\ct{2.4e7}} es $2.4\times 10^7$.
	\index{n\'umero en coma flotante}

\item[Caracteres] Un signo d\'olar introduce un \subind{literal}{car\'acter} literal: \ct{$a}\ignoredollar$ es el literal para `a'.
  Pueden obtenerse instancias de caracteres no imprimibles enviando los mensajes adecuados a la clase \clsind{Character}, como \ct{Character space}\cmindex{clase Character}{space} y \ct{Character tab}\cmindex{clase Character}{tab}.
		
\item[Cadenas de caracteres] Las comillas simples se utilizan para definir una \subind{literal} cadena literal.
  Si quieres una cadena que contenga una comilla, simplemente se usan dos comillas, como en \ct{'entre ''comillas'' simples'}.
\luq{ejemplo en lugar de 'G''day' ?}\nicopaez{te la debo}

\item[S\'imbolos] se parecen a las cadenas en que contienen una secuencia de caracteres.
	Sin embargo, al contrario que en una cadena, se garantiza que un s\'imbolo \subind{s\'imbolo}{literal} es \'unico globalmente.
	S\'olo existe un objeto s\'imbolo \ct{#Hola}, pero puede haber varios objetos String con el valor \ct{'Hola'}.
		\seeindex{\#@{\textsf{\#}}}{s\'imbolo literal}

\item[Arrays en tiempo de compilaci\'on] Se definen mediante \ct{#( )} rodeando literales separados por espacios.
  Dentro de los par\'entesis todo debe ser constante en tiempo de compilaci\'on.
  Por ejemplo, \ct{#(27 (true false) abc)} es un \subind{literal}{array} \subind{array}{literal} de tres elementos: el entero \ct{27}, el array en tiempo de compilaci\'on conteniendo los dos booleanos y el s\'imbolo \ct{#abc} (observa que esto es lo mismo que \ct{#(27 #(true false) #abc)}).

\item[Arrays en tiempo de ejecuci\'on] Las llaves \ct|{ }| definen un array (\subind{Array}{din\'amico}) en tiempo de ejecuci\'on.
  Sus elementos son expresiones separadas por puntos.
  Por lo tanto, \ct|{ 1. 2. 1+2 }| define un array con los elementos 1, 2 y el resultado de evaluar 1+2
  (la notaci\'on de llaves es espec\'ifica de los dialectos de \st \pharo y \squeak. En otros \st{}s tienes que construir expl\'icitamente los arrays din\'amicos).

\item[Comentarios] Se encierran entre comillas dobles.
		\ct{"Hola"} es un \ind{comentario}, no una cadena de caracteres, y es ignorado por el compilador de \pharo.
		Los comentarios pueden ocupar varias l\'ineas.

\item[Definiciones de variables locales] Las barras verticales \ct{| |} encierran la \subind{variable}{declaraci\'on} de una o m\'as variables locales en un m\'etodo (y tambi\'en en un bloque).
		\seeindex{declaraci\'on}{declaraci\'on de variable}

\item[Asignaci\'on] \ct{:=} asigna un objeto a una variable.
%		Sometimes you will see $\leftarrow$ used instead.
%		Unfortunately, since this is not an \textsc{ascii} character, it will appear as an underscore unless
%		you are using a special font.
%		So, \ct{x := 1} is the same as \ct{x _ 1} or \ct{x UNDERSCORE 1}. You should use \ct{:=} since the other representations have been deprecated.
		\index{asignaci\'on}
		\seeindex{:=@{\textsf{:=}}}{asignaci\'on}
		\seeindex{\_@{\textsf{\_}}}{asignaci\'on}
		\seeindex{<-@{$\leftarrow$}}{asignaci\'on}

\item[Bloques] Los corchetes \ct{[ ]} definen un \ind{bloque}, conocido tambi\'en como clausura de bloque o clausura l\'exica, que es un objeto de primer orden representando una funci\'on. Como veremos, los bloques pueden tener argumentos y variables locales.
	\seeindex{[ ]@{\textsf{[ ]}}}{bloque}
	\seeindex{clausura}{bloque}
	\seeindex{clausura l\'exica}{bloque}

\item[Primitivas] \ct{<primitive: ...>} indica la invocaci\'on de una \ind{primitiva} de \ind{m\'aquina virtual.}
	(\ct{<primitive: 1>} es la primitiva de m\'aquina virtual para \ct{SmallInteger>>>+}).
	Cualquier c\'odigo que siga a la primitiva es ejecutado s\'olo si la primitiva falla.
	La misma sintaxis se utiliza tambi\'en para las anotaciones de los m\'etodos.

\item[Mensajes unarios] consisten en una sola palabra (como \ct{factorial}) enviada a un receptor (como \ct{3}).
	\index{mensaje!unario}
	\seeindex{mensaje unario}{mensaje, unario}

\item[Mensajes binarios] son operadores (como \ct{+}) enviados a un receptor y que tienen un \'unico argumento. En \ct{3+4}, el receptor es \ct{3} y el argumento es \ct{4}.
	\index{mensaje!binario}
	\seeindex{mensaje binario}{mensaje, binario}

\item[Mensajes de palabra clave] consisten en varias palabras clave (como \ct{raisedTo:modulo:}), acabando cada una con dos puntos y recibiendo un \'unico argumento.
En la expresi\'on \ct{2 raisedTo: 6 modulo: 10}, el \emphind{selector del mensaje} \ct{raisedTo:modulo:} recibe los dos argumentos \ct{6} y \ct{10}, cada uno siguiente los dos puntos. Enviamos el mensaje al receptor \ct{2}.
	\index{mensaje!palabra clave}
	\seeindex{mensaje de palabra clave}{mensaje, palabra clave}

\item[Retorno de m\'etodo] \ct{^} se usa para \emphind{retornar} un valor desde un m\'etodo (debes escribir \verb|^| para obtener el caracter \ct{^}).\nicopaez{ojo que me parece que esto ya no aplica en pharo}

\item[Secuencias de sentencias]	Un punto (\ct{.}) es el \emphsubind{sentencia}{separador} de \emph{sentencias}. Poner un punto entre dos expresiones las convierte en sentencias independientes.
	\seeindex{punto}{separador de sentencia}
	\seeindex{\ct{.}}{separador de sentencia}

\item[Cascadas] Los puntos y coma pueden usarse para enviar una \emphind{cascada} de mensajes a un \'unico receptor.  En \ct{Transcript show: 'hola'; cr} primero enviamos el mensaje de palabra clave \ct{show: 'hola'} al receptor \ct{Transcript} y entonces enviamos el mensaje unario \ct{cr} al mismo receptor.
	\seeindex{;}{cascada}

\end{description}

Las clases \ct{Number}, \ct{Character}, \ct{String} y \ct{Boolean} se describen con m\'as detalle en \charef{basic}.\nicopaez{hay que ver como esto de las referencias}

%=================================================================
\section{Pseudo-variables}

En \st existen 6 palabras reservadas o \emph{pseudo-variables}:
\pvind{nil}, \pvind{true},  \pvind{false},  \pvind{self}, \pvind{super} y \pvind{thisContext}.
Se llaman  \subind{variable}{pseudo}-variables porque est\'an predefinidas y no pueden ser asignadas.
\ct{true}, \ct{false} y \ct{nil} son constantes, mientras que los valores de \ct{self}, \ct{super} y \ct{thisContext} var\'ian din\'amicamente cuando el c\'odigo es ejecutado.

\ct{true} y \ct{false} son las \'unicas instancias de las clases \clsind{Boolean} \clsind{True} y \clsind{False}.
V\'ease \charef{basic} para m\'as detalles.

\pvind{self} siempre se refiere al receptor del m\'etodo ejecut\'andose actualmente.

\ct{super} tambi\'en se refiere al receptor del mensaje actual, pero cuando env\'ias un mensaje a \super, la b\'usqueda del m\'etodo cambia, de modo que comienza desde la superclase de la clase que contiene el m\'etodo que usa \ct{super}.
Para m\'as detalles v\'ease \charef{model}.

\ct{nil} es el objeto indefinido.
Es la \'unica instancia de la clase \clsind{UndefinedObject}. 
Las variables de instancia, variables de clase y variables locales se inicializan a \ct{nil}.

\ct{thisContext} es una pseudo-variable que representa el marco superior de la pila de ejecuci\'on.
En otras palabras, representa el \clsind{MethodContext} o \clsind{BlockClosure} ejecut\'andose actualmente.
Normalmente, \ct{thisContext} no es de inter\'es para la mayor\'ia de los programadores, pero es imprescindible para implementar herramientas de desarrollo como el depurador y tambi\'en se utiliza para implementar el manejo de excepciones y las continuaciones.

%=================================================================
\section{Env\'ios de mensajes}

Existen tres tipos de mensajes en \pharo.
\begin{enumerate}
  \item Los mensajes \emph{unarios} no reciben ning\'un argumento.
  \ct{1 factorial} env\'ia el mensaje \ct{factorial} al objeto \ct{1}.
  \item Los mensajes \emph{binarios} reciben exactamente un argumento.
  	\ct{1 + 2} env\'ia el mensaje \ct{+} con el argumento \ct{2} al objeto \ct{1}.
  \item Los mensajes \emph{de palabra clave} reciben un n\'umero arbitrario de argumentos.
  	\ct{2 raisedTo: 6 modulo: 10} env\'ia el mensaje que consiste en el selector del mensaje
	\ct{raisedTo:modulo:} y los argumentos \ct{6} y \ct{10} al objeto \ct{2}.
\end{enumerate}

Los selectores de los mensajes unarios consisten en caracteres alfanum\'ericos y comienzan con una letra min\'uscula.
\index{mensaje!unario}

Los selectores de los mensajes binarios consisten en uno o m\'as de los siguientes caracteres:
\index{mensaje!binario}
\begin{code}{}
+ - / \ * ~ < > = @ % | & ! ? ,
\end{code}

Los selectores de mensajes de palabra clave consisten en una serie de palabras clave alfanum\'ericas que comiezan con una letra min\'uscula y acaban con dos puntos.
\index{mensaje!de palabra clave}

Los mensajes unarios tienen la mayor precedencia, despu\'es los mensajes binarios y por \'ultimo, los mensajes de palabra clave, por lo tanto:
\begin{code}{@TEST}
2 raisedTo: 1 + 3 factorial --> 128
\end{code}
(Primero enviamos \ct{factorial} a \ct{3}, despu\'es enviamos \ct{+ 6} a \ct{1} y por \'ultimo, enviamos \ct{raisedTo: 7} a \ct{2}.)  
Recuerda que usamos la notaci\'on \lct{\emph{expresi\'on}}\ct{-->}\lct{\emph{resultado}} para mostrar el resultado de la evaluaci\'on de una expresi\'on.

Aparte de la precedencia, la evaluaci\'on se \nicopaez{me parece que en lugar de 'se' deberia ser 'es'} estrictamente de izquierda a derecha, con lo que:
\begin{code}{@TEST}
1 + 2 * 3 --> 9
\end{code}
no \ct{7}.
Se deben usar par\'entesis para cambiar el orden de la evaluaci\'on:
\begin{code}{@TEST}
1 + (2 * 3) --> 7
\end{code}

Los env\'ios de mensajes pueden componerse por puntos y puntos y coma. Una secuencia de expresiones separadas por puntos causa que cada expresi\'on sea evaluada como una \emphind{sentencia} detr\'as de otra.
\index{sentencia!separador}

\begin{code}{}
Transcript cr.
Transcript show: 'hola mundo'.
Transcript cr
\end{code}

\noindent
Esto enviar\'a \ct{cr} al objeto \glbind{Transcript}, despu\'es le enviar\'a \ct{show: 'hello world'} y por \'ultimo, otro \ct{cr}.

Cuando una serie de mensajes se env\'ia al \emph{mismo} receptor, puede expreserse m\'as brevemente como una \emphind{cascada}.
El receptor se especifica s\'olo una vez y la secuencia de mensajes se separa por puntos y coma:

\begin{code}{}
Transcript cr;
    show: 'hola mundo';
    cr
\end{code}
Esto tiene exactamente el mismo efecto que el ejemplo anterior.

%=================================================================
\section{Sintaxis de los m\'etodos}

Mientras que las expresiones se pueden evaluar en cualquier sitio en \pharo (por ejemplo, en un espacio de trabajo, en un depurador o en un navegador), los m\'etodos se definen normalmente en una ventana de navegador o en el depurador
(los m\'etodos tambi\'en pueden ser introducidos desde un medio externo, pero esta no es la forma habitual de programar en \pharo).

Los programas se desarrollan escribiendo m\'etodo a m\'etodo en el contexto de una clase dada (una clase se define enviando un mensaje a una clase existente, pidi\'endole crear una clase, por lo que no es necesaria una sintaxis especial para la definici\'on de clases).

Este es el m\'etodo \mthind{String}{lineCount} en la clase \clsind{String} (el convenio habitual es referirse a los m\'etodos como \ct{NombreDeClase>>>nombreDeMetodo}, por lo que llamamos a este m\'etodo \ct{String>>>lineCount}).

\needlines{9}
\begin{method}[lineCount]{Line count}
String>>>lineCount
   "Answer the number of lines represented by the receiver,
   where every cr adds one line."
   | cr count |
   cr := Character cr.
   count := 1 min: self size.
   self do:
      [:c | c == cr ifTrue: [count := count + 1]].
   ^ count
\end{method}

Sint\'acticamente un m\'etodo consiste en:
\begin{enumerate}
  \item El patr\'on del met\'odo, que contiene el nombre (es decir, \ct{lineCount}) y los argumentos (ninguno en este ejemplo);
  \item comentarios (pueden aparecer en cualquier sitio, pero el convenio es colocar uno al comienzo que explique lo que hace el m\'etodo);
  \item declaraciones de las variables locales (es decir, \ct{cr} y \ct{count}); y
  \item cualquier n\'umero de expresiones separadas por puntos; aqu\'i hay cuatro.
\end{enumerate}

La evaluaci\'on de cualquier expresi\'on precedida por un \ct{^} (escrito como \verb|^|) causar\'a que el m\'etodo salga en ese punto, devolviendo el valor de la expresi\'on.
Un m\'etodo que termina sin devolver expl\'icitamente ninguna expresi\'on, devolver\'a impl\'icitamente \pvind{self}.
\index{return!implicit}

Los argumentos y variables locales deber\'ian comenzar siempre con letras min\'usculas.
Los nombres que comienzan con letras may\'usculas se supone que son variables globales.
Los nombres de clases, como \ct{Character}, por ejemplo, son simplemente variables globales que se refieren al objeto que representa esa clase.


%=================================================================
\section{Sintaxis de los bloques}

Los bloques proporcionan un mecanismo para diferir la evaluaci\'on de expresiones.
Un \ind{bloque} es b\'asicamente una funci\'on an\'onima. Un bloque se eval\'ua envi\'andole el mensaje \mthind{BlockClosure}{value}.
El bloque responde con el valor de la \'ultima expresi\'on de su cuerpo, si no existe un valor de retorno expl\'icito (con \ct{^}), en cuyo caso no responde ning\'un valor.
\seeindex{value}{BlockClosure}

\begin{code}{@TEST}
[ 1 + 2 ] value --> 3
\end{code}

Los bloques pueden recibir par\'ametros, cada uno de los cuales se declara con dos puntos precedi\'endole. Una barra vertical separa la declaraci\'on de par\'ametros del cuerpo del bloque. Para evaluar un cuerpo con un par\'ametro, debes enviar el mensaje \mthind{BlockClosure}{value:} con un argumento. A un bloque de dos par\'ametros se le debe enviar \mthind{BlockClosure}{value:value:}, y as\'i, hasta los 4 argumentos.

\begin{code}{@TEST}
[ :x | 1 + x ] value: 2 --> 3
[ :x :y | x + y ] value: 1 value: 2 --> 3
\end{code}

Si tienes un bloque con m\'as de cuatro par\'ametros, debes usar \mthind{BlockClosure}{valueWithArguments:} y pasar los argumentos en un array (un bloque con un n\'umero grande de par\'ametros es con frecuencia se\~nal de un problema de dise\~no).

Los bloques pueden tambi\'en declarar variables locales, que son repodeadas\nicopaez{querras decir 'rodeadas'?} por barras verticales, tal como las declaraciones de variables locales en un m\'etodo.
Las variables locales son declaradas despu\'es de los argumentos:
\index{variable!declaraci\'on}

\begin{code}{@TEST}
[ :x :y | | z | z := x+ y. z ] value: 1 value: 2 --> 3
\end{code}

Los bloques son en realidad \emph{cierres} l\'exicos, ya que pueden referirse a las variables del entorno circundante. El siguiente bloque se refiere a la varaible \ct{x} del entorno que lo contiene:

\begin{code}{@TEST}
| x |
x := 1.
[ :y | x + y ] value: 2 --> 3
\end{code}

Los bloques son instancias de la clase \clsind{BlockClosure}.
Esto significa que son objetos, por lo que pueden ser asignados a variables y pasados como argumentos como cualquier otro objeto.

% For both understandability and performance, it is better for blocks to refer only to their parameters and local variables; blocks that do not refer external variables are optimized by the compiler.
% MARCUS sez: I would just delete the sentence. There is nothing optimized, accessign outer temps is as fast as inner, so the only reason to avoid accessing outer temps would be that the code is easier to understand. But that's a relatively weak argument, I think.
% However, the ability to refer (``capture'') non-local variables can be very powerful when it is needed. 

%\paragraph{Really important.} \^\ acts as an escaping mechanism. 
%Return expressions inside a nested block expression will terminate the enclosing method.
%In the example 

%\begin{script}[detect]{...} when the expression \ct{^\ x@y} is executed, the method \ct{detect:}
% escapes the current iteration and returns it. 

%TwoLevelSet>>detect: aBlock

%   firstLevel keysAndValuesDo: [ :x :v |
%      v do: [ :y | (aBlock value: x@y) ifTrue: [^x@y]]
%   ].
%   ^nil
%\end{script}


%=================================================================
\section{Condicionales y bucles en dos palabras} \nicopaez{tal vez aqui deberiamos mantener el mismo criterio que usamo para el titulo del capitulo}

\st no ofrece una sintaxis especial para construcciones de control. En su lugar, t\'ipicamente se expresan mediante el env\'io de mensajes a booleanos, n\'umeros y colecciones, con bloques como argumentos.

Los condicionales se expresan enviando uno de los mensajes \mthind{Boolean}{ifTrue:}, \mthind{Boolean}{ifFalse:} o \mthind{Boolean}{ifTrue:ifFalse:} al resultado de una expresi\'on booleana. V\'ease \charef{basic} para m\'as sobre booleanos.

\begin{code}{}
(17 * 13 > 220)
   ifTrue: [ 'mayor' ]
   ifFalse: [ 'menor' ] --> 'mayor'
\end{code}
% ON: Not a test.
% My regex approach cannot handle multi-line expressions :-(

Los bucles son t\'ipicamente expresados mediante el env\'io de mensajes a bloques, enteros o colecciones.
Puesto que la condici\'on de salida para un bucle puede ser evaluada repetidamente, deber\'ia ser un bloque en lugar de un valor booleano.
Este es un ejemplo de un bucle muy procedimental:
\index{iteraci\'on}
\index{iteraci\'on|seealso{Colecci\'on, iteraci\'on}}
\seeindex{bublces}{iteraci\'on}
\seeindex{enumeraci\'on}{iteraci\'on}
\seeindex{construcciones de control}{iteraci\'on}

\begin{code}{@TEST | n |}
n := 1.
[ n < 1000 ] whileTrue: [ n := n*2 ].
n --> 1024
\end{code}
\cmindex{BlockClosure}{whileTrue:}

\noindent
\mthind{BlockClosure}{whileFalse:} invierte la condici\'on de salida.
\begin{code}{@TEST | n |}
n := 1.
[ n > 1000 ] whileFalse: [ n := n*2 ].
n --> 1024
\end{code}

\noindent
\mthind{Integer}{timesRepeat:} ofrece una forma sencilla de implementar un iteraci\'on fija:

\begin{code}{@TEST | n |}
n := 1.
10 timesRepeat: [ n := n*2 ].
n --> 1024
\end{code}

Tambi\'en podemos enviar el mensaje \mthind{Number}{to:do:} a un n\'umero que entonces act\'ua como el valor inicial del contador de un bucle.
Los dos argumentos son el l\'imite superior y un bloque que toma el valor actual del contador del bucle como su argumento: 

\needlines{4}
\begin{code}{@TEST | result |}
result := String new.
1 to: 10 do: [:n | result := result, n printString, ' '].
result --> '1 2 3 4 5 6 7 8 9 10 '
\end{code}

\damien{I think the previous example that I've just added is clearer than the one which is commented here.}
% \begin{code}{@TEST | n |}
% n := 0.
% 1 to: 10 do: [ :counter | n := n + counter ].
% n --> 55
% \end{code}

\paragraph{Iteradores de orden superior.}
Las colecciones comprenden un gran n\'umero de clases diferentes, muchas de las cuales soportan el mismo protocolo.
Los mensajes m\'as importantes para iterar sobre las colecciones incluyen 
\mthind{Collection}{do:}, \mthind{Collection}{collect:}, \mthind{Collection}{select:}, \mthind{Collection}{reject:}, \mthind{Collection}{detect:} and \mthind{Collection}{inject:into:}.
Estos mensajes definen iteradores de orden superior que nos permiten escribir c\'odigo muy compacto.

Un \clsind{Interval} es una colecci\'on que nos permite iterar sobre una secuencia de n\'umeros desde el punto de inicio hasta el final. \ct{1 to: 10} representa el intervalo desde 1 hasta 10.
Ya que es una colecci\'on, podemos enviarle el mensaje \ct{do:}.
El argumento es un bloque que es evaluado por cada elemento de la colecci\'on.

\begin{code}{@TEST | result |}
result := String new.
(1 to: 10) do: [:n | result := result, n printString, ' '].
result --> '1 2 3 4 5 6 7 8 9 10 '
\end{code}

\damien{Again, I think the previous example is clearer than the one which is commented here.}
% \begin{code}{@TEST | n |}
% n := 0.
% (1 to: 10) do: [ :element | n := n + element ].
% n --> 55
% \end{code}

\ct{collect:} construye una nueva colecci\'on del mismo tama\~no, transformando cada elemento.
\begin{code}{@TEST}
(1 to: 10) collect: [ :each | each * each ] --> #(1 4 9 16 25 36 49 64 81 100)
\end{code}

\ct{select:} y \ct{reject:} construyen nuevas colecciones, cada \nicopaez{me parece que aqui deberia ser 'cada una'} consistiendo en un subconjunto de los elementos que satisfacen (o no) la condici\'on del bloque booleano.
\ct{detect:} devuelve el primer elemento que satisface la condici\'on.
No olvides que las cadenas son tambi\'en colecciones, por lo que puedes iterar sobre todos los caracteres.

\begin{code}{@TEST}
'hello there' select: [ :char | char isVowel ] --> 'eoee'
'hello there' reject: [ :char | char isVowel ] --> 'hll thr'
'hello there' detect: [ :char | char isVowel ] --> $e
\end{code}

Finalmente, deber\'ias tener presente que las colecciones tambi\'en soportan un operador como el \emph{fold} del estilo funcional en el m\'etodo \ct{inject:into:} method \nicopaez{me parece que esta ultima palabra esta de sobra}.
Este te \nicopaez{quisiste poner 'te' o se te escapo?} permite generar un resultado acumulado usando una expresi\'on que comienza con un valor semilla y que inyecta cada elemento de la colecci\'on.
Sumas y productos son ejemplo t\'ipicos.
\seeindex{fold}{\ct{Collection>>>inject:into}}

\begin{code}{@TEST}
(1 to: 10) inject: 0 into: [ :sum :each | sum + each ] --> 55
\end{code}

\noindent
Esto es equivalente a \ct{0+1+2+3+4+5+6+7+8+9+10}.

Puedes encontrar m\'as sobre colecciones en  \charef{collections}.

%=================================================================
\section{Primitivas y pragmas}

En \st todo es un objeto y todo ocurre mediante el env\'io de mensajes.
No obstante, en ciertos puntos tocamos el fondo.
Algunos objetos s\'olo pueden funcionar invocando las \ind{primitiva}{}s de la \ind{m\'aquina virtual}.

Por ejemplo, todo lo siguiente es implementado como primitivas:
reserva de memoria (\mthind{Behavior}{new}, \mthind{Behavior}{new:}),
manipulaci\'on de bits (\mthind{Integer}{bitAnd:}, \mthind{Integer}{bitOr:}, \mthind{Integer}{bitShift:}),
aritm\'etica entera y decimal (\ct{+}, \ct{-},  \ct{<},  \ct{>}, \ct{*}, \ct{/ }, \ct{=}, \ct{==}...),
y el acceso a los arrays (\mthind{Object}{at:}, \mthind{Object}{at:put:}).
\seeindex{new@{\ct{new}}}{\ct{Behavior>>>new}}

Las primitivas se invocan con la sintaxis \ct{<primitive: unNumero>}.
Un m\'etodo que invoque tal primitiva tambi\'en puede incluir c\'odigo \st, que ser\'a evaluado \emph{s\'olo} si la primitiva falla.

A continuaci\'on vemos le c\'odigo para \cmind{SmallInteger}{+}.
Si la primitiva falla, la expresi\'on \ct{super + unNumero} ser\'a evaluada y devuelta.

\needlines{6}
\begin{method}[primitive]{Un m\'etodo primitivo}
+ aNumber 
  "Primitive. Add the receiver to the argument and answer with the result
  if it is a SmallInteger. Fail if the argument or the result is not a
  SmallInteger  Essential  No Lookup. See Object documentation whatIsAPrimitive."

  <primitive: 1>
  ^ super + aNumber
\end{method}

%The other use of primitives is to optimize some crucial methods. The idea is that the system could work 
%without the primitive but it would be slow. The following method shows that the method \ct{@} is calling the primitive 18. Here the point creation is clearly expressible in \st therefore the code after the primitive is just the creation of a point illustrating what the primitive is actually doing. Note that such a code will be never called except if the primitive would failed which is extremely rare.  

%\begin{method}[xxx]{xxx}
%Integer>>@ y 
%   "Primitive. Answer a Point whose x value is the receiver and whose y 
%   value is the argument. Optional. No Lookup. See Object documentation 
%   whatIsAPrimitive."

%   <primitive: 18>
%   ^Point x: self y: y
%\end{method}


En \pharo, la sintaxis del corchete angular tambi\'en se utiliza para las anotaciones a los m\'etodos, llamadas pragmas.
\sd{we should give an example}\ab{Please do!  Is don't know about these.}\damien{it's the third time we talk about pragmas without saying what they are and how to use them.}

%=================================================================
\section{Resumen del cap\'itulo}

\begin{itemize}

\item	\pharo (s\'olo) tiene) \nicopaez{estos parentesis no me cierran, hay algo mal} seis identificadores reservados tambi\'en llamados \textit{pseudo-variables}: \ct{true}, \ct{false}, \ct{nil}, \ct{self}, \ct{super} y \ct{thisContext}.

\item	Hay cinco tipos de objetos literales: n\'umeros (\ct{5}, \ct{2.5}, \ct{1.9e15}, \ct{2r111}), caracteres (\ct{$a}), cadenas (\ct{'hola'}), s\'imbolos (\ct{#hola}) y arrays (\ct{#('hola' #hola)})

\item	Las cadenas est\'an delimitadas por comillas simples, los comentarios por comillas dobles.
		Para tener una cita dentro de una cadena, usa doble comilla.

\item	A diferencia de las cadenas, se garantiza que los s\'imbolos son \'unicos globalmente.

\item	Usa \ct{#( ... )} para definir un array literal.
		Usa \ct|{ ... }| para definir un array din\'amico.
		Observa que
		\ct{#( 1 + 2 ) size --> 3}, pero que
		\ct|{ 1 + 2 } size --> 1|

\item	Hay tres tipos de mensajes:
		\emph{unarios} (\eg \ct{1 asString}, \ct{Array new}),
		\emph{binarios} (\eg \ct{3 + 4}, \ct{'hola' , 'mundo'}) y
		\emph{de palabra clave} (\eg \ct{'hola' at: 2 put: $o})

\item	Un mensaje \emph{en cascada} env\'ia una secuencia de mensajes al mismo destino \nicopaez{me suena mejor 'destinatario'}, separados por puntos y coma:
\ct{OrderedCollection new add: #calvin; add: #hobbes; size --> 2}

\item	Las variables locales se declara con barras verticales.
		Usa \ct{:=} para la asignaci\'on.
		\ct{|x| x:=1}

\item	Las expresiones consisten en env\'ios de mensajes, cascadas y asignaciones, posiblemente agrupados con par\'entesis.
		Las \emph{sentencias} son expresiones separadas por puntos.

\item	Los bloques de cierre son expresiones encerradas en corchetes.
		Los bloques pueden tener argumentos y pueden contener variables temporales.
		Las expresiones en el bloque no se eval\'uan hasta que env\'ias al bloque un mensaje
		\ct{value...} con el n\'umero de argumentos correcto.\\
		\ct{[:x | x + 2] value: 4 --> 6}.

\item	No hay una sintaxis dedicada para las construcciones de control, s\'olo mensajes que condicionalmente eval\'uan bloques.\\
		\ct{(\st includes: Class) ifTrue: [ Transcript show: Class superclass ]}

\end{itemize}

%=================================================================
\ifx\wholebook\relax\else
\end{document}\fi
%=================================================================
%%% Local Variables:
%%% coding: utf-8
%%% mode: latex
%%% TeX-master: t
%%% TeX-PDF-mode: t
%%% ispell-local-dictionary: "spanish"
%%% End:
