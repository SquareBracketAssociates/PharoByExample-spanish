% $Author$
% $Date$
% $Revision$

%Gabi: Testeo para ver si funciona el commit de rluque

% HISTORY:
% 2006-10-05 - Oscar started
% 2007-05-28 - Stef edit
% 2007-06-06 - Oscar first draft
% 2007-08-14 - Stef corrections
% 2007-09-06 - Lukas review
% 2009-08-12 - Oscar rewrite for Pharo

%=================================================================
\ifx\wholebook\relax\else
% --------------------------------------------
% Lulu:
	\documentclass[a4paper,spanish, 10pt,twoside]{book}
	\usepackage[
		papersize={6.13in,9.21in},
		hmargin={.75in,.75in},
		vmargin={.75in,1in},
		ignoreheadfoot
	]{geometry}
	\input{../common.tex}
	\pagestyle{headings}
	\setboolean{lulu}{true}
% --------------------------------------------
% A4:
%	\documentclass[a4paper,11pt,twoside]{book}
%	\input{../common.tex}
%	\usepackage{a4wide}
% --------------------------------------------
    \graphicspath{{figures/} {../figures/}}
	\begin{document}
	% \renewcommand{\nnbb}[2]{} % Disable editorial comments
	\sloppy
	\frontmatter
\fi
%=================================================================
\chapter{Prefacio}\chalabel{intro}

%=================================================================
\section*{¿Qué es \pharo?}

\pharo es una implementación completa, moderna y de código abierto, del lenguaje y ambiente de programación \st. \pharo es un derivado de \squeak\cite{Inga97a}, una reimplementación del clásico sistema \st-80. Mientras que \squeak fue desarrollado principalmente como una plataforma para desarrollo experimental de software educacional, \pharo se esfuerza en ofrecer una plataforma limpia y de código abierto para desarrollo profesional de software y una plataforma robusta y estable para investigación y desarrollo en lenguajes y ambientes dinámicos. \pharo sirve como la implementación de referencia para el framework de desarrollo web Seaside.

\pharo resuelve algunos temas de licenciamiento de \squeak. A diferencia de versiones anteriores de \squeak, el núcleo de \pharo contiene solo código que ha sido contribuido bajo licencia MIT. EL proyecto \pharo comenzó en Marzo de 2008 como una bifurcación de \squeak 3.9, y la primera versión beta 1.0 fue liberada el 31 de Julio de 2009.


Aunque \pharo remueve muchos paquetes de \squeak, también incluye numerosas características que son opcionales en \squeak. Por ejemplo, las fuentes true type estan incluidas en \pharo. \pharo también incluye soporte para block closures. Las interfaces de usuario han sido simplificadas y revisadas.

\pharo es altamanento portable --- incluso su máquina virtual está escrita enteramente en \st, haciendo fácil su depuración, análisis y cambio. \pharo es el vehículo para un ámplio rango de proyectos innovadores desde aplicaciones multimedia y plataformas educacionales hasta ambientes comerciales de desarrollo web. 

Hay un aspecto importante detrás de \pharo: \pharo no debe ser sólo una copia del pasado sino verdaderamente \emph{reinventar} Smalltalk. Enfoque Big-bang raramente triunfan. \pharo favorecerá  cambios evolutivos e incrementales. Queremos ser capaces de experimentar con características o bibliotecas nuevas importantes. Evolución significa que \pharo acepta equivocaciones y no es su objetivo ser la solución perfecta en un sólo paso grande\,---\,incluso aunque nos encantaria. \pharo favorecerá cambios incrementales pequeños pero una multitud de ellos. El éxito de \pharo depende de las contribuciones de su comunidad.
% The \pharo community will pay attention to your submissions to improve the system.

%=================================================================
\section*{¿Quién debería leer este libro?}

Este libro está basado en \emph{Squeak by Example}\footnote{\sbe}, una introducción de código abierto a \squeak.
El libro ha sido libremente adaptado y revisado para reflejar las differencias entre \pharo y \squeak.
El libro presenta varios aspectos de \pharo, comenzando con lo básico y avanzando a temas más avanzados.

El libro no enseña cómo programar. El lector debería tener cierta familiaridad con lenguajes de programación. Algún conocimiento de programación orientada a objetos también sería de ayuda.

El libro presenta el ambiente de programación \pharo, el lenguaje y las herramientas asociadas. Serás expuesto a modismos y prácticas comunes, pero el foco está en la tecnologia, no en el diseño orientado a objetos. Cuando sea posible, te mostraremos muchos ejemplos. (Hemos sido inspirados por el excelente libro de Smalltalk\cite{Shar97a} de Alec Sharp.)
\index{Sharp, Alex}

Hay numerosos otros libros  de \st gratuitamente disponibles en la web pero ninguno de ellos se enfoca específicamente en \pharo. Por ejemplo:
\url{http://stephane.ducasse.free.fr/FreeBooks.html}

\ifluluelse{}{\newpage} % layout hint
%=================================================================
\section*{A word of advice}

% http://www.surfscranton.com/architecture/KnightsPrinciples.htm

Do not be frustrated by parts of \st that you do not immediately understand.
You do not have to know everything!
Alan Knight expresses this principle as follows\footnote{\url{http://www.surfscranton.com/architecture/KnightsPrinciples.htm}}:
\index{Knight, Alan}
\important{{\bf Try not to care.}
Beginning \st programmers often have trouble because they think they need to understand all the details of how a thing works before they can use it. This means it takes quite a while before they can master \ct{Transcript show: 'Hello World'}. One of the great leaps in OO is to be able to answer the question ``How does this work?'' with ``I don't care''.}

%=================================================================
\section*{An open book}

This book is an open book in the following senses: 

\begin{itemize}

\item	The content of this book is released under the Creative Commons Attribution-ShareAlike (by-sa) license.
		In short, you are allowed to freely share and adapt this book, as long as you respect the conditions of the license available at the following URL: 
		\url{http://creativecommons.org/licenses/by-sa/3.0/}.

\item	This book just describes the core of \pharo.
		Ideally we would like to encourage others to contribute chapters
		on the parts of \pharo that we have not described.
		If you would like to participate in this effort, please
		contact us.  We would like to see this book grow!
\end{itemize}

For more details, visit \pbe.

%=================================================================
\section*{The \pharo community}

The \pharo community is friendly and active.
Here is a short list of resources that you may find useful:

\begin{itemize}
\item \url{http://www.pharo-project.org} is the main web site of \pharo.
%environment built on top of \pharo but whose audience is elementary
%school teachers.) % I remove this [Martial: french contributor]

\item \url{http://www.squeaksource.com} is the equivalent of SourceForge for \pharo projects.
Many optional packages for \pharo live here.
\end{itemize}

%=================================================================
\section*{Examples and exercises}

We make use of two special conventions in this book.

We have tried to provide as many examples as possible.
In particular, there are many examples that show a fragment of code which can be evaluated.  We use the symbol \ct{-->} to indicate the result that you obtain when you select an expression and \menu{print it}:

\begin{code}{@TEST}
3 + 4 --> 7    "if you select 3+4 and 'print it', you will see 7"
\end{code}

In case you want to play in \pharo with these code snippets, you can download a plain text file with all the example code from the book's web site: \pbe.

The second convention that we use is to display the icon \dothisicon{} to indicate when there is something for you to do:

\dothis{Go ahead and read the next chapter!}

%=================================================================
\section*{Acknowledgments}

We would first like to thank the original developers of \squeak for making this amazing \st development environment available as an open source project.

% We would like to thank various people who have contributed to this book.
% In particular, we thank
We would also like to thank Hilaire Fernandes and Serge Stinckwich who allowed us to translate parts of their columns on \st, and Damien Cassou for contributing the chapter on streams.

We especially thank Alexandre Bergel, Orla Greevy, Fabrizio Perin, Lukas Renggli, Jorge Ressia and Erwann Wernli for their detailed reviews.

We thank the University of Bern, Switzerland, for graciously supporting this open-source project and for hosting the web site of this book.

We also thank the Squeak community for their enthusiastic support of this book project, and for informing us of the errors found in the first edition of this book.

%=============================================================
\ifx\wholebook\relax\else
   \bibliographystyle{jurabib}
   \nobibliography{scg}
   \end{document}
\fi
%=============================================================
