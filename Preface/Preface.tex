% $Author$
% $Date$
% $Revision$

%Gabi: Testeo para ver si funciona el commit de rluque

% HISTORY:
% 2006-10-05 - Oscar started
% 2007-05-28 - Stef edit
% 2007-06-06 - Oscar first draft
% 2007-08-14 - Stef corrections
% 2007-09-06 - Lukas review
% 2009-08-12 - Oscar rewrite for Pharo

%=================================================================
\ifx\wholebook\relax\else
% --------------------------------------------
% Lulu:
	\documentclass[a4paper,spanish, 10pt,twoside]{book}
	\usepackage[
		papersize={6.13in,9.21in},
		hmargin={.75in,.75in},
		vmargin={.75in,1in},
		ignoreheadfoot
	]{geometry}
	\input{../common.tex}
	\pagestyle{headings}
	\setboolean{lulu}{true}
% --------------------------------------------
% A4:
%	\documentclass[a4paper,11pt,twoside]{book}
%	\input{../common.tex}
%	\usepackage{a4wide}
% --------------------------------------------
    \graphicspath{{figures/} {../figures/}}
	\begin{document}
	% \renewcommand{\nnbb}[2]{} % Disable editorial comments
	\sloppy
	\frontmatter
\fi
%=================================================================
\chapter{Prefacio}\chalabel{intro}

%=================================================================
\section*{¿Qué es \pharo?}

\pharo es una implementación completa, moderna y de código abierto, del lenguaje y ambiente de programación \st. \pharo es un derivado de \squeak\cite{Inga97a}, una reimplementación del clásico sistema \st-80. Mientras que \squeak fue desarrollado principalmente como una plataforma para desarrollo experimental de software educacional, \pharo se esfuerza en ofrecer una plataforma limpia y de código abierto para desarrollo profesional de software y una plataforma robusta y estable para investigación y desarrollo en lenguajes y ambientes dinámicos. \pharo sirve como la implementación de referencia para el framework de desarrollo web Seaside.

\pharo resuelve algunos temas de licenciamiento de \squeak. A diferencia de versiones anteriores de \squeak, el núcleo de \pharo contiene solo código que ha sido contribuido bajo licencia MIT. EL proyecto \pharo comenzó en Marzo de 2008 como una bifurcación de \squeak 3.9, y la primera versión beta 1.0 fue liberada el 31 de Julio de 2009.


Aunque \pharo remueve muchos paquetes de \squeak, también incluye numerosas características que son opcionales en \squeak. Por ejemplo, las fuentes true type estan incluidas en \pharo. \pharo también incluye soporte para block closures. Las interfaces de usuario han sido simplificadas y revisadas.

\pharo es altamanento portable --- incluso su máquina virtual está escrita enteramente en \st, haciendo fácil su depuración, análisis y cambio. \pharo es el vehículo para un ámplio rango de proyectos innovadores desde aplicaciones multimedia y plataformas educacionales hasta ambientes comerciales de desarrollo web. 

Hay un aspecto importante detrás de \pharo: \pharo no debe ser sólo una copia del pasado sino verdaderamente \emph{reinventar} Smalltalk. Enfoque Big-bang raramente triunfan. \pharo favorecerá  cambios evolutivos e incrementales. Queremos ser capaces de experimentar con características o bibliotecas nuevas importantes. Evolución significa que \pharo acepta equivocaciones y no es su objetivo ser la solución perfecta en un sólo paso grande\,---\,incluso aunque nos encantaria. \pharo favorecerá cambios incrementales pequeños pero una multitud de ellos. El éxito de \pharo depende de las contribuciones de su comunidad.
% The \pharo community will pay attention to your submissions to improve the system.

%=================================================================
\section*{¿Quién debería leer este libro?}

Este libro está basado en \emph{Squeak by Example}\footnote{\sbe}, una introducción de código abierto a \squeak.
El libro ha sido libremente adaptado y revisado para reflejar las differencias entre \pharo y \squeak.
El libro presenta varios aspectos de \pharo, comenzando con lo básico y avanzando a temas más avanzados.

El libro no enseña cómo programar. El lector debería tener cierta familiaridad con lenguajes de programación. Algún conocimiento de programación orientada a objetos también sería de ayuda.

El libro presenta el ambiente de programación \pharo, el lenguaje y las herramientas asociadas. Serás expuesto a modismos y prácticas comunes, pero el foco está en la tecnologia, no en el diseño orientado a objetos. Cuando sea posible, te mostraremos muchos ejemplos. (Hemos sido inspirados por el excelente libro de Smalltalk\cite{Shar97a} de Alec Sharp.)
\index{Sharp, Alex}

Hay numerosos otros libros  de \st gratuitamente disponibles en la web pero ninguno de ellos se enfoca específicamente en \pharo. Por ejemplo:
\url{http://stephane.ducasse.free.fr/FreeBooks.html}

\ifluluelse{}{\newpage} % layout hint
%=================================================================
\section*{Un consejo}

% http://www.surfscranton.com/architecture/KnightsPrinciples.htm

No te fustres por aquellas partes de \st que no entiendas inmediatamente.
¡No tienes que saber todo!
Alan Knight expresa este principio de la siguiente manera: \footnote{\url{http://www.surfscranton.com/architecture/KnightsPrinciples.htm}}:
\index{Knight, Alan}
\important{{\bf Try not to care.}
Los programadores \st principiantes amenudo tienen problemas porque piensan que necesitan entender todos los detalles de como una funciona antes de poder usarla. Esto significa que les lleva un rato antes de que puedan dominar el \ct{Transcript show: 'Hello World'}. One of the great leaps in OO is to be able to answer the question ``How does this work?'' with ``I don't care''.}

%=================================================================
\section*{Un libro abierto}

Este libro es un libro abierto en el siguiente sentido:

\begin{itemize}

\item	El contenido de este libro es publicado bajo licencia Creative Commons Attribution-ShareAlike (by-sa).
		Resumiendo, puedes compartir y adaptar este libro, siempre que respetes las condiciones de la licencia disponibles en la siguiente URL: 
		\url{http://creativecommons.org/licenses/by-sa/3.0/}.

\item	Este libro describe sólo el núcleo de de \pharo.
		Idealmente nos gustaría movitar a otro para que contribuyan con capítulos
		de otras partes de \pharo que no hemos descripto.
		Si quieres participar en este esfuerzo, por favor contáctanos. ¡Nos gustaria ver crecer este libro!
\end{itemize}

Para más detalles, visita \pbe.

%=================================================================
\section*{La comunidad \pharo}

La comunidad \pharo es amistosa y activa.
Este es una corta lista de recursos que puedes encontrar útil:

\begin{itemize}
\item \url{http://www.pharo-project.org} is the main web site of \pharo.
%environment built on top of \pharo but whose audience is elementary
%school teachers.) % I remove this [Martial: french contributor]

\item \url{http://www.squeaksource.com} is the equivalent of SourceForge for \pharo projects.
Many optional packages for \pharo live here.
\end{itemize}

%=================================================================
\section*{Ejemplos y ejercicios}

Usamos dos convenciones especiales en este libro.

Hemos intentado proveer tantos ejemplos como nos fue posible.
En particular, hay muchos ejemplos que muestran un fragmento de código que puede ser evaluado. Utilizamos el símbolo \ct{-->} para indicar el resultado que obtienes cuando seleccionas una expresión y ejecutas \menu{print it}:

\begin{code}{@TEST}
3 + 4 --> 7    "if you select 3+4 and 'print it', you will see 7"
\end{code}

En caso que quieras jugar en \pharo con estos fragmentos de código, puedes descargar un archivo de texto plano con todo el código de ejemplo desde el sitio web del libro: \pbe.

La segunda convención que usamos es mostrar el ícono \dothisicon{} para indicar cuando hay algo para que hagas:

\dothis{Go ahead and read the next chapter!}

%=================================================================
\section*{Acknowledgments}

We would first like to thank the original developers of \squeak for making this amazing \st development environment available as an open source project.

% We would like to thank various people who have contributed to this book.
% In particular, we thank
We would also like to thank Hilaire Fernandes and Serge Stinckwich who allowed us to translate parts of their columns on \st, and Damien Cassou for contributing the chapter on streams.

We especially thank Alexandre Bergel, Orla Greevy, Fabrizio Perin, Lukas Renggli, Jorge Ressia and Erwann Wernli for their detailed reviews.

We thank the University of Bern, Switzerland, for graciously supporting this open-source project and for hosting the web site of this book.

We also thank the Squeak community for their enthusiastic support of this book project, and for informing us of the errors found in the first edition of this book.

%=============================================================
\ifx\wholebook\relax\else
   \bibliographystyle{jurabib}
   \nobibliography{scg}
   \end{document}
\fi
%=============================================================
